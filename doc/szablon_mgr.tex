% SZABLON PRACY MAGISTERSKIEJ - WERSJA 1.0 Z 14 LUTEGO 2010 - MARCIN KULCZYCKI
% Szablon wymaga LaTeXa z zainstalowanymi polskimi stylami
% W skład szablonu wchodzą pliki szablon_mgr.tex i logo_uj.png
% Plik należy kompilować bezpośrednio do formatu .pdf, nie do .dvi (inaczej grafika nie złoży się dobrze)

% Uwaga! Po każdej większej zmianie oraz przed ostatecznym drukiem plik należy przeLaTeXować trzy razy pod
% rząd, aby referencje, cytowania oraz spis treści miały szansę złożyć się we właściwy sposób.

\documentclass[12pt,a4paper,oneside]{book}

\usepackage[utf8]{inputenc}
\usepackage{amssymb} % To pakiet z dodatkowymi symbolami matematycznymi
\usepackage{polski} % Te pakiety umożliwiają składanie pracy w języku polskim
\usepackage{graphicx} % Ten pakiet umożliwia umieszczanie obrazków w tekście
\usepackage{indentfirst}	
\usepackage{subfigure}
\usepackage{hyperref}
\usepackage{enumitem}
\usepackage{listings}
\usepackage{anysize}
\usepackage{setspace}

%marginesy
	\marginsize{3.5cm}{2.5cm}{2.5cm}{2.5cm}

%	\numberwithin{figure}{chapter}  % numerowanie obrazków, prefiksowane numerem sekcji
%	\numberwithin{table}{chapter}   % numerowanie tabel, prefiksowane numerem sekcji
%	\numberwithin{section}{chapter}

%interlinia
	\doublespacing
	
%ustawienia linków (brak obramowania)
	\hypersetup{
	pdfborder={0 0 0 0} 
	}

%brak dzielenia wyrazów
%	\selecthyphenation{nohyphenation}
%	\sloppy


%ustawienia listingów
\lstset{ %
language=XML,                % choose the language of the code
basicstyle=\footnotesize,       % the size of the fonts that are used for the code
numbers=left,                   % where to put the line-numbers
numberstyle=\footnotesize,      % the size of the fonts that are used for the line-numbers
stepnumber=1,                   % the step between two line-numbers. If it's 1 each line will be numbered
numbersep=5pt,                  % how far the line-numbers are from the code
showspaces=false,               % show spaces adding particular underscores
showstringspaces=false,         % underline spaces within strings
showtabs=false,                 % show tabs within strings adding particular underscores
tabsize=2,	                % sets default tabsize to 2 spaces
captionpos=b,                   % sets the caption-position to bottom
breaklines=true,                % sets automatic line breaking
breakatwhitespace=false,        % sets if automatic breaks should only happen at whitespace
title=\lstname,                 % show the filename of files included with \lstinputlisting; also try caption instead of title
escapeinside={\%*}{*)}          % if you want to add a comment within your code
}

\begin{document}

% Strona tytułowa
\begin{center}
\thispagestyle{empty} % Na stronie tytułowej nie chcemy numeru strony
\includegraphics[width=1.5cm, height=1.5cm]{logo_uj.png} % Ta komenda umieszcza na stronie logo UJ

{\large UNIWERSYTET JAGIELLOŃSKI

WYDZIAŁ FIZYKI, ASTRONOMII I INFORMATYKI STOSOWANEJ} \vfill\vfill\vfill\vfill

{\large TOMASZ BOROWSKI \bigskip}

{\Huge SZTUCZNA INTELIGENCJA
W~SYMULATORZE DZIAŁAŃ
ANTYTERRORYSTYCZNYCH}\bigskip\bigskip

{\large PRACA MAGISTERSKA NAPISANA POD KIERUNKIEM

DR HAB. PIOTRA BIAŁASA \vfill\vfill\vfill

KRAKÓW 2012} 
\end{center}


% Streszczenie
\chapter*{Streszczenie}
Niniejsza praca dyplomowa omawia projekt gry symulacyjnej, w~której gracz ma możliwość planowania i przeprowadzania działań antyterrorystycznych. Zastosowane w projekcie algorytmy sztucznej inteligencji, typowe dla gier wideo, zostały uzupełnione algorytmami realizującymi charakterystyczne dla strony konfliktu taktyki. Dokumentacja projektu jest uzupełniona opisem technologi HTML5 Canvas oraz bibliotek JavaScript wykorzystanych podczas implementacji.


% Spis treści
\newpage
\phantomsection \label{tocChapter}
\addcontentsline{toc}{chapter}{Spis treści}
\tableofcontents % Spis treści jest generowany automatycznie przez LaTeXa
\pagestyle{plain} % Ta komenda pozwala pozbyć się nagłówków stron

% Oświadczenie
\newpage
\phantomsection \label{oswiadczenieChapter}
\addcontentsline{toc}{chapter}{Oświadczenie}
\chapter*{Oświadczenie}
Świadomy odpowiedzialności prawnej oświadczam, że złożona praca magisterska pt.: „Sztuczna inteligencja w symulatorze działań antyterrorystycznych” została napisana przeze mnie samodzielnie.

Równocześnie oświadczam, że praca ta nie narusza prawa autorskiego w rozumieniu ustawy z dnia 4 lutego 1994 roku o prawie autorskim i prawach pokrewnych (Dz.U.1994 nr 24 poz. 83) oraz dóbr osobistych chronionych prawem cywilnym.

Ponadto praca nie zawiera informacji i danych uzyskanych w sposób nielegalny i~nie była wcześniej przedmiotem innych procedur urzędowych związanych z uzyskaniem dyplomów lub tytułów zawodowych uczelni wyższej.



% Wstęp
\newpage
\phantomsection \label{wprowadzenieChapter}
\addcontentsline{toc}{chapter}{Wprowadzenie}
\chapter*{Wprowadzenie}
Gry wideo, które dotychczas kojarzone były niemal wyłącznie z pojęciem interaktywnej formy dostarczania rozrywki, od wielu lat zdobywają coraz to nowsze pola zastosowań. Przykładem tutaj mogą być gry oparte o zasadę tzw. \emph{edutainment} (w~tłum. \emph{edurozrywka})\footnote{przykładowy serwis z grami edukacyjnymi - http://www.edugames.pl/}. Mają one na celu efektywne przekazywanie wiedzy, dzięki swojemu atrakcyjnemu i rozrywkowemu charakterowi, w takich dyscypliach naukowych jak biologia, fizyka, informatyka lub języki obce. 
Innym polem zastosowań elementów gier jest biznes. Coraz częściej można spotkać się z pojęciem \emph{gamefication} (w tłum. \emph{grywalizacji}) miejsca pracy. Określa ono zestaw technik i narzędzi związanych z grami, które pomagają motywować pracowników do lepszego wykonywania powierzonej im pracy. Dzieje się to poprzez nagradzanie najlepszych pracowników wirtualnymi punktami doświadczenia, osiągnięciami oraz umieszczaniem ich wizerunku na szczytach rankingów\footnote{przykładowa aplikacja bazująca na idei grywalizacji - https://dueprops.com/}.
Wreszcie, możemy mieć również do czynienia z~grami symulacyjnymi. Ich celem jest umożliwienie graczom doznawania wrażeń znanych z~rzeczywistości, a których oni bezpośrednio mogą na codzień nie doświadczać. Wśród takich gier można wyróżnić gry, których celem jest szkolenie użytkowników - np. symulatory lotu - oraz te, których głównym celem jest dostarczenie użytkownikom rozrywki - np. symulator prowadzenia sieci pizzerii.

\begin{figure}
\begin{center}
	\includegraphics[width=120mm,height=90mm]{images/flightSim}
	\caption{Fligt Simulator 2004 - przykład gry symulacyjnej}
\end{center}
\end{figure}

Niniejsza praca dyplomowa skupia się na projekcie gry symulacyjnej, która odwzorowuje, w dużym uproszczeniu, działania oddziałów antyterrorystyczych podczas szturmu na budynek, zajęty przez wrogie jednostki. Użytkownik grający w tą grę ma możliwość stworzenia schematu budynku, parametryzacji liczby jednostek po obu stronach konfliktu oraz określenia planu działania antyterrorystów. Na podstawie tej konfiguracji gra przeprowadza symulację szturmu na budynek, którą gracz może obserwować.

Realizacja tego projektu obejmuje zaprojektowanie i zaimplementowanie gry oraz omówienie taktyk stosowanych przez strony konfliktu. Zwrócona jest szczególna uwaga na twórcze wykorzystaie algorytmów sztucznej inteligencji, charakterystycznych dla gier wideo. Uzupełnieniem dokumentu jest przedstawienie technologii i bibliotek, które zostały wykorzystane podczas implementacji.




% Rozdział 1
\chapter{State of Art}
\section{Planowanie operacji antyterrorystycznych w~rzeczywistości}\label{realChapter}
Problem terroryzmu i skutków, jakie może on wyrządzać ludności, jest dla instytucji państwowych podstawą do przygotowywania długoterminowych strategii jego zapobiegania. Strategie te ujęte są w dokumentach\footnote{polskim przykładem jest dokument "Narodowy Program Antyterrorystyczny RP na lata 2012-2016"} przygotowywanych przez instrumenty państwowe. Opisują one środki i metody zabezpieczania obywateli przed aktami terroryzmu. Niestety, w zetknięciu z rzeczywistością bywają one nie zawsze skuteczne.

Mając do czynienia z aktem terroryzmu, polegającym na przejęciu kontroli przez terrorystów nad pewną przestrzenią (np. nad budynkiem), służby odpowiadające za bezpieczeństwo podejmują szereg działań, które mają na celu zminimalizować ryzyko utraty zdrowia lub życia przez osoby postronne (w tym ew. zakładników). Prócz zabezpieczenia okolicznego terenu (odizolowaniu go od cywili oraz mediów) oraz prowadzenia negocjacji z terrorystami, bardzo ważnym elementem jest przygotowanie planu przejęcia zakładników oraz ew. eliminacji terrorystów z użyciem siły. Do takiej czynności może dojść w przypadku, gdy terroryści odmówią negocjacji, bądź gdy zaczynają zabijać zakładników.

Proces planowania akcji antyterrorystycznych jest często charakterystyczny dla przeprowadzającej go jednostki specjalnej i zawsze jest strzeżony tajemnicą. Jednakże na przełomie kwietnia i maja 1980 roku, gdy grupa sześciu terrorystów przejęła kontrolę nad Ambasadą Irańską w Londynie, biorąc za zakładników 26 osób, to brytyjskie jednostki specjalne przeprowadziły skuteczną eliminację terrorystów na oczach całego świata\footnote{świadkami operacji byli dziennikarze wielu stacji telewizyjnych, a wśród zakładników byli m. in. reporterzy BBC}. Dzisiaj Operacja Nimrod jest szczegółowo udokumentowana licznymi artykułami\footnote{przy przygotowywaniu tej pracy został wykorzystany artykuł ze strony Elite UK Forces\cite{eliteUK}}, książkami oraz dokumentami wideo. Dzięki tej wiedzy jesteśmy w stanie odtworzyć proces planowania takiej akcji antyterrorystycznej, co zostało ukazane w tabeli \ref{realPlan}. Spełnienie wszystkich wymienionych czynności znacznie zwiększa szanse na powodzenie operacji: uratowanie zakładników, eliminacja terrorystów i nieodniesienie strat własnych przez jednostkę przeprowadzającą atak.

\begin{table}
\begin{center}
\begin{tabular}{p{0.5\textwidth} p{0.5\textwidth}}
Planowana czynność & Realizacja (Nimrod)\\\hline
	\begin{enumerate}
		\setlength\itemsep{0pt}
		\item Przygotowanie IA Plan\footnote{Immediate Action Plan - plan eliminacji terrorystów, który jest przygotowywany przed powstaniem docelowego planu (najczęściej w tym czasie nie ma jeszcze danych wywiadowczych)}
		\item Zbieranie danych
		\item Rozpoznanie wroga
		\item Rozpoznanie wyposażenia wroga
		\item Rozpoznanie terenu
		\item Określenie niezbędnych środków 
		\item Określenie punktów wejścia
		\item Określenie punktów ewakuacji		
	\end{enumerate}&\begin{enumerate}
		\setlength\itemsep{0pt}
		\item Szturm ambasady od głównego wejścia i zabezpieczanie budynku piętro po piętrze
		\item Zainstalowane podsłuchy w ścianach, snajperzy jako obserwatorzy, sprawdzanie punktów wejścia pod osłoną nocy
		\item Wywiad dostarcza dane osobowe terrorystów, którzy starali się o~wizy w ambasadzie Wielkiej Brytanii w Belgradzie
		\item Jeden z uwolnionych zakładników informuje policję o liczbie i uzbrojeniu terrorystów
		\item Analizowane są plany architektoniczne budynku i prowadzona jest konsultacja z woźnym ambasady
		\item Cztery drużyny (24 żołnierzy), pistolety maszynowe MP5, ładunki wybuchowe, granaty ogłuszające, liny itp.
		\item Wejście przez dach, wejście przez balkony na pierwszym piętrze, wejście tylnymi drzwiami na parterze
		\item Ewakuacja zakładników do ogrodu za budynkiem ambasady
	\end{enumerate}
\end{tabular}
\caption {Czynności dokonywane podczas planowania operacji antyterrorystycznej\label{realPlan}}
\end{center}
\end{table} 

W grze symulacyjnej, będącej przedmiotem tej pracy dyplomowej, gracz może zaplanować podstawowe elementy operacji antyterrorystycznej:
\begin{enumerate}
	\item zdefiniować liczbę terrorystów i antyterrorystów
	\item zaplanować jednopoziomową architekturę budynku
	\item oznaczyć punkty kluczowe wokół których można spodziewać się obecności terrorystów
	\item zdefiniować punkt wejścia oraz punkt ewakuacji
\end{enumerate}

\section{Gry symulacyjne}
Gatunek gier symulacyjnych charakteryzuje się wiernym odzwierciedlaniem realiów świata rzeczywistego lub fikcyjnego. Prócz zastosowania rozrywkowego, gry symulacyjne wykorzystuje się do celów szkoleniowych (np. wirtualna nauka jazdy) lub badawczych (np. analiza bezpieczeństwa terytorialnego). Wśród symulacyjnych gier wideo należy wymieć kilka podgatunków\footnote{przedstawiona lista wywodzi się z podziału przedstawionego w książce A. Rollingsa i E. Adamsa\cite{gameDesign} i dopełniona jest podgatunkami omawianymi w różnych publikacjach internetowych}:

\begin{description}
	\item[Symulatory budowania i zarządzania] cechują się brakiem obecności wroga, którego gracz musi pokonać. Są to gry o pewnych procesach (ekonomicznych, politycznych, wytwórczych itp.), w ramach których gracz odgrywa rolę architekta i zarządcy. Obiektami budowanymi mogą być parki rozrywki, porty lotnicze, szpitale, zoo czy też miasta. Im lepiej gracz rozumie zachodzące procesy, tym skuteczniejszy jest w wykonywaniu powierzonych mu zadań. Pierwszym symulatorem tego typu była gra \textbf{SimCity} [Maxis 1989].
	\item[Symulatory życia] pozwalają na kontrolowanie istnień i rozwijaniu relacji między nimi. Mechanizmy są tu podobne do symulatorów budowania i zarządzania, często nie ma określonego kryterium zwycięstwa. Gry symulacyjne, gdzie gracz hoduje zwierzę lub jakiś antropomorficzny twór, skupiają się na tworzeniu i~rozwijaniu relacji tej formy życia z graczem. Przykładami takich gier jest \textbf{The Sims} [Maxis 2000] oraz \textbf{Spore} [Maxis 2008].
	\item[Symulatory sportowe] pozwalają graczowi na wirtualne uprawianie dyscyplin sportowych, których zasady i kryteria zwycięstwa są zgodne z rzeczywistymi odpowiednikami\footnote{choć część zasad może być wyłączana, np. czas trwania meczu piłkarskiego lub błąd kroków w~koszykówce}. Często takie symulatory wymagają od swoich twórców modelowania rzeczywistych postaci ze świata sportu, wraz z uwzględnieniem ich umiejętności, charakterystycznych ruchów czy ubioru. Przykładami takich gier są gry z serii \textbf{Pro Evolution Soccer} [Konami] oraz \textbf{NBA Live} [EA Sports].
	\item[Symulatory pojazdów] mają na celu dostarczyć graczom wrażeń, jakie mogliby odczuć podczas kierowania rzeczywistymi pojazdami, w określonych warunkach. Tego typu gry najcześciej charakteryzują się bardzo wysoką wiernością odzwierciedlenia pojazdów, do której należy zaliczyć takie czynniki jak wygląd, parametry jazdy lub lotu, wyposażenie oraz sterowanie. Przykładami takich gier jest seria \textbf{Colin McRae Rally} [Codemasters] oraz seria \textbf{Microsoft Flight Simulator} [Microsoft].
	\item[Symulatory czynności i zawodów] to dość popularny w ostatnim czasie typ gier. Mają one na celu umożliwienie graczom na wirtualne wykonywanie prac związanych z zawodami, którymi na co dzień się nie zajmują. Przykładami takich gier jest \textbf{Symulator Farmy 2011} [Atari / Infogrames 2011] czy \textbf{Symulator Koparki 2011} [astragon Software 2011]. Realizm nie jest tutaj najważniejszym kryterium.
\end{description}

Grę symulacyjną, będącą przedmiotem tej pracy dyplomowej, można sklasyfikować w podgatunku symulatorów czynności i zawodów.

\section{Sztuczna inteligencja w grach}\label{aiModelInfo}
Sztuczna inteligencja, jako dział informatyki, zajmuje się analizą zachowań człowieka oraz formalizowaniem (np. w postaci algorytmów) zaobserwowanych procesów m.in. myślowych i decyzyjnych. Dzięki takiej analizie jest możliwe przygotowywanie programów, pozwalających na rozwiązywanie problemów, które do tej pory były domeną ludzką. Przykładami mogą tu być wyszukiwanie danych, rozpoznawanie obiektów, syntezacja mowy lub podejmowanie decyzji. W tym celu algorytmy sztucznej inteligencji mogą wykorzystywać implementacje takich zagadnień jak sieci neuronowe, algorytmy genetyczne czy logika rozmyta.

W grach wideo sztuczna inteligencja najczęściej sprowadza się do zastosowania prostych technik sztucznej inteligencji, które mają na celu zaspokoić trzy podstawowe potrzeby bohaterów gry\cite{aiForGames}:
\begin{itemize}
	\item zdolność poruszania się 
	\item zdolność do podejmowania decyzji gdzie należy się poruszyć
	\item zdolność taktycznego i strategicznego myślenia
\end{itemize}

\begin{figure}
\begin{center}
	\includegraphics[width=120mm,height=66mm]{images/pacman}
	\caption{Pac-Man - przykład prostych technik sztucznej inteligencji w grach\label{pacman}}
\end{center}
\end{figure}

\textbf{Pac-Man} [Namco, 1980] była jedną z pierwszych gier, która posiadała zauważalne dla odbiorców elementy sztucznej inteligencji. Gracz, poruszając się po dwuwymiarowym labiryncie, zdobywał punkty zjadając kropki (rysunek \ref{pacman}). W tej czynności aktywnie przeszkadzały mu cztery duchy, które starały się podążać korytarzami labiryntu w kierunku gracza. Od strony implementacyjnej gra opierała się o bardzo prostą maszynę stanową, która dla duchów definiowała dwa stany: podążaj za graczem i uciekaj od gracza. Na każdym skrzyżowaniu dróg labiryntu podejmowana była decyzja\footnote{decyzja była losowa lub poparta prostymi obliczeniami} o następnym kierunku.

W późniejszych grach elementy myślenia i podejmowania decyzji stawały się coraz bardziej rozbudowane. Przykładem jest gra \textbf{Goldeneye 007} [Rare Ltd. 1997], gdzie postaci zostały wyposażone w system symulowanych zmysłów. Jedna postać analizowała pulę informacji ze świata gry, co pozwalało np. na dostrzeżenie martwego towarzysza i wykonanie odpowiedniej reakcji na ten fakt, czyli zmiany własnego stanu.

\begin{figure}
\begin{center}
	\includegraphics[width=120mm,height=80mm]{images/aimodel}
	\caption{AI Model - zdefiniowany przez I. Millingtona i J. Funge\label{aimodel}}
\end{center}
\end{figure}

Analizując elementy składowe sztucznej inteligencji w grach wideo, należy odwołać się do modelu AI (rysunek \ref{aimodel}). Postaci z gry posiadają wiedzę (całościową lub cząstkową) o świecie w którym funkcjonują (\emph{World interface}). Na podstawie tej wiedzy każda postać, za pomocą odpowiednich algorytmów, podejmuje jakieś decyzje (\emph{Decision making}) oraz porusza się (\emph{Movement}). Element strategii (\emph{Strategy}) jest przetwarzany na poziomie grupy postaci i może wpływać na podejmowane przez jednostki decyzje lub wykonywane ruchy. Rezultatem bezpośrednim tych obliczeń są wykonywane animacje (\emph{Animation}) oraz wyliczenia fizyki ruchu postaci (\emph{Physics}). Efektem ubocznym mogą tu być zmiany stanu gry, polegające na modyfikacji elementów świata (\emph{Content creation}) gry oraz wykonywaniu oskryptowanych akcji (\emph{Scripting}).

W grze symulacyjnej, będącą przedmiotem tej pracy dyplomowej, będziemy mogli wyróżnić każdy z trzech elementów modelu sztucznej inteligencji:
\begin{description}
	\item[Podejmowanie decyzji] np. otwarcie ognia do wroga
	\item[Poruszanie się] np. poruszanie po ścieżce, wędrowanie
	\item[Strategia] np. role i ich przejmowanie w grupie antyterrorystów
\end{description}

\section{Istniejące rozwiązania: Tom Clancy's Rainbow Six}

Gra \textbf{Rainbow Six} [Red Storm 1998] bazuje na powieści Toma Clancy'ego o~tym samym tytule, która opisuje działania tajnego, międzynarodowego oddziału antyterrorystycznego Rainbow. Gra łączy w sobie elementy FPP\footnote{First Person Perspective - gra akcji z pierwszoosobową perspektywą} oraz strategii. Przeprowadzane operacje antyterrorystyczne są każdorazowo poprzedzane planowaniem szturmu na podstawie mapy lokacji (rysunek \ref{rainbowPlan}).

\begin{figure}
\begin{center}
	\includegraphics[width=120mm,height=90mm]{images/planner}
	\caption{Tom Clancy's Rainbow Six - planowanie operacji antyterrorystycznej\label{rainbowPlan}}
\end{center}
\end{figure}

Przed misją gracz otrzymuje dane wywiadowcze, które posiadają zbliżony charakter do tych, które są wykorzystywane w planowaniu w rzeczywistości (szczegóły rozdziale \ref{realChapter}). Podczas planowania misji w Rainbow Six gracz może:
\begin{enumerate}
	\item zdefiniować liczbę drużyn antyterrorystów
	\item zdefiniować liczbę antyterrorystów w drużynie oraz wskazać ich wyposażenie	
	\item oznaczyć punkty kluczowe, wzdłuż których będzie poruszać się drużyna antyterrorystów
	\item zdefiniować lokacje, w których drużyna będzie czekała na polecenia innej drużyny
\end{enumerate}

Po zaplanowaniu operacji antyterrorystycznej, gracz może uczestniczyć aktywnie w rozgrywce (będąc dowódcą jednej z drużyn) lub przyglądać się jej w roli obserwatora. Celem misji jest najczęściej odbicie zakładników, ale może też nim być rozbrojenie ładunków wybuchowych, zdobycie danych lub eliminacja konkretnej postaci. Element planowania misji wyróżnia Rainbow Six spośród innych gier o podobnej tematyce. Gra okazała się na tyle popularna, że doczekała się kolejnych części.

Różnice pomiędzy planowaniem operacji antyterrorystycznej w Rainbow Six a~grą symulacyjną, będącą przedmiotem tej pracy dyplomowej, wymieniono w tabeli~\ref{diffTab}

\begin{table}
\begin{center}
\begin{tabular}{p{0.5\textwidth} p{0.5\textwidth}}
Rainbow Six & Gra symulacyjna\\\hline
	\begin{enumerate}
		\setlength\itemsep{0pt}
		\item Brak definiowania lokacji
		\item Posida podział antyterrorystów na drużyny	
		\item Szczegółowe definiowanie wyposażenia 	
		\item Możliwość podglądu planu w 3D			
	\end{enumerate}&\begin{enumerate}
		\setlength\itemsep{0pt}
		\item Posiada prosty edytor lokacji
		\item Dostępna jest tylko jedna drużyna antyterrorystów
		\item Brak możliwości definiowania wyposażenia
		\item Podgląd planu wyłącznie w 2D
	\end{enumerate}
\end{tabular}
\caption {Różnice pomiędzy planowaniem w Rainbow Six a przygotowaną grą symulacyjną\label{diffTab}}
\end{center}
\end{table} 

\section{HTML5 Canvas i Kinetic.js}
\textbf{Canvas} to część języka \textbf{HTML5}, której początki sięgają 2004 roku. Pozwala ona na dynamiczne renderowanie kształtów oraz obrazów w obrębie dokumentu HTML. Dzięki temu tworzenie animacji 2D i 3D nie wymaga instalowania dodatkowego oprogramowania, ponieważ całość jest obsługiwana przez środowisko współczesnych przeglądarek internetowych\footnote{zgodność danej przeglądarki internetowej ze standardami HTML5 można sprawdzić pod adresem http://html5test.com/}.

Istnieje duża ilość bibliotek javascript'owych, które ułatwiają pracę z HTML5 Canvas. Jedną z nich jest \textbf{Kinetic.js}, która dodatkowo pozwala na animowanie obiektów na scenie, przetwarzanie ich (translacje, rotacje, skalowanie itp.) oraz obsługę zdarzeń. Scena w Kinetic.js jest złożona z warstw zdefiniowanych przez użytkownika. Każda warstwa składa się z dwóch kontekstów: kontekst sceny i kontekst bufora. Podczas gdy kontekst sceny reprezentuje to, co jest renderowane na ekranie, to kontekst bufora odpowiada za wydajną obsługę zdarzeń. Każda warstwa może zawierać kształty lub grupy kształtów, które mogą być indywidualnie lub grupowo przetwarzane.

W grze symulacyjnej, będącej przedmiotem tej pracy dyplomowej, zostały zastosowane następujące biblioteki javascript'owe:
\begin{itemize}
	\item Kinetic.js\cite{kineticPage} - pozwala na renderowanie obiektów oraz ich przetwarzanie
	\item Sylvester\cite{sylvesterPage} - pozwala na obliczenia na wektorach 
	\item jQuery\cite{jqueryPage} - pozwala na przetwarzanie elementów HTML
	\item javascript-astar\cite{astarPage} - pozwala na wyszukiwanie ścieżek algorytmem A*
\end{itemize}


% Rozdział 2
\chapter{Założenia projektu}
\section{Wymagania funkcjonalne}
Realizacja projektu opierała się w całości o stosowanie tzw. technik zwinnych\footnote{Agile development - punktem wyjścia do tego podejścia jest Manifest Zwinnego Tworzenia Oprogramowania z 2001 roku http://agilemanifesto.org/iso/pl/}. Proces tworzenia gry symulacyjnej został podzielony na etapy. Przed implementacją każdego etapu przygotowywany był zestaw scenariuszy opisujący funkcjonalności, jakie powinny zostać zaimplementowane w danym etapie. Natomiast po implementacji każdego z etapów gra symulacyjna była udostępniania kilku testerom, którzy w ramach informacji zwrotnej wskazywali, jakie funkcjonalności lub zachowania jednostek chcieliby zaobserwować w grze. Przy czytaniu scenariuszy przydatna jest znajomość słownika pojęć projektu (rozdział \ref{dictonary}).

Pierwszy etap implementacji projektu zakładał zbudowanie architektury kodu aplikacji - utworzenie podstawowych klas, metod odpowiedzialnych za zarządzanie obiektami na scenie oraz metody wyznaczania bezkolizyjnej ścieżki do zadanego punktu. Dodatkowo jednostki miały mieć możliwość poruszania się do określonego punktu docelowego. Szczegóły zostały przedstawione w tabeli \ref{scenarios1}.

\begin{table}
\begin{center}
\begin{tabular}{|p{1.0\textwidth}|}
\hline
Etap 1 - Setup aplikacji\\\hline
	\begin{itemize}
		\setlength\itemsep{0pt}
		\item aplikacja może tworzyć obiekty i renderować je na scenie
		\item aplikacja może tworzyć terrorystów (obiekty ruchome)
		\item aplikacja może tworzyć antyterrorystów (obiekty ruchome)
		\item aplikacja może tworzyć ściany (obiekty statyczne)
		\item aplikacja może wyliczać ścieżki dla obiektów ruchomych
		\item jednostki mogą się poruszać do zadanego punktu	
	\end{itemize}
\\\hline
\end{tabular}
\caption {Zestaw scenariuszy dla funkcjonalności pierwszego etapu\label{scenarios1}}
\end{center}
\end{table} 

Podczas drugiego etapu miały zostać zaimplementowane kluczowe elementy interfejsu użytkownika. Miał on pozwalać na skonfigurowanie symulacji poprzez zbudowanie ścian, określenie punktów kluczowych oraz zdefiniowanie liczby uczestniczących jednostek. Ponadto należało przygotować odpowiednie kontrolki, które sterują symulacją. Szczegóły zostały przedstawione w tabeli \ref{scenarios2}. 

\begin{table}
\begin{center}
\begin{tabular}{|p{1.0\textwidth}|}
\hline
ETAP 2 - Interfejs\\\hline
	\begin{itemize}
		\setlength\itemsep{0pt}
		\item interfejs pozwala na definiowanie ścian (dodawanie nowej, usuwanie ostatniej, usuwanie wszystkich)
		\item interfejs pozwala na definiowanie punktu startowego / końcowego antyterrorystów (dodawanie, usuwanie)
		\item interfejs pozwala na definiowanie punktów kluczowych (dodawanie nowego, usuwanie ostatniego, usuwanie wszystkich)
		\item interfejs pozwala na definiowanie liczby antyterrorystów oraz liczby terrorystów
		\item interfejs pozwala na rozpoczęcie symulacji
		\item interfejs pozwala na zakończenie symulacji
		\item interfejs pozwala na wstrzymanie symulacji
	\end{itemize}
\\\hline
\end{tabular}
\caption {Zestaw scenariuszy dla funkcjonalności drugiego etapu\label{scenarios2}}
\end{center}
\end{table} 

Implementacja trzeciego etapu zakładała wdrożenie podstawowych elementów taktyk dla jednostek. Domyślnym zachowaniem terrorystów jest wędrowanie, które może być losowo wstrzymywane na kilka sekund. Domyślnym zachowaniem antyterrorystów jest podążanie w małych odstępach jeden za drugim, za wyjątkiem lidera, który podąża wytyczoną ścieżką do kolejnych punktów kluczowych. Ponadto implementacja zakładała wdrożenie systemu logów - generowanie wiadomości dotyczących kluczowych momentów w symulacji. Szczegóły zostały przedstawione w tabeli \ref{scenarios3}.

\begin{table}
\begin{center}
\begin{tabular}{|p{1.0\textwidth}|}
\hline
ETAP 3 - Poruszanie się\\\hline
	\begin{itemize}
		\setlength\itemsep{0pt}
		\item interfejs może wyświetlać logi dotyczące aktualnej symulacji
		\item antyterrorysta będący liderem może poruszać się ścieżką po punktach kluczowych
		\item antyterrorysta nie będący liderem może poruszać się w linii za poprzedzającym go antyterrorystą
		\item terrorysta może wędrować
		\item terrorysta może stać
	\end{itemize}
\\\hline
\end{tabular}
\caption {Zestaw scenariuszy dla funkcjonalności trzeciego etapu\label{scenarios3}}
\end{center}
\end{table} 

Czwarty etap implementacji projektu zakładał wprowadzenie elementu walki między jednostkami. Jednostka może zaatakować wrogą jednostkę wystrzeliwując pociski. Pociski trafiające w jednostkę zmniejszają jej liczbę punktów życia przeciw proporcjonalnie do odległości, jaką pokonał wystrzelony pocisk (symulacja utraty energii). Gdy liczba punktów życia danej jednostki spada poniżej zera, wtedy ta jednostka ginie. Ponadto postrzelona jednostka próbuje podążać do lokacji, z której padł strzał. Jeżeli w walce polegnie lider antyterrorystów, to jego funkcję (prowadzenie grupy) przejmuje następny antyterrorysta. Dodatkowo w~tym etapie miały zostać zaimplementowane nowe funkcjonalności interfejsu, które pozwalają użytkownikowi na zapisywanie, usuwanie oraz wczytywanie wcześniej przygotowanej konfiguracji. Szczegóły zostały przedstawione w tabeli \ref{scenarios4}.

\begin{table}
\begin{center}
\begin{tabular}{|p{1.0\textwidth}|}
\hline
ETAP 4 - Odczyt / zapis oraz walka\\\hline
	\begin{itemize}
		\setlength\itemsep{0pt}
		\item interfejs pozwala na zapisanie bieżącej konfiguracji
		\item interfejs pozwala na usunięcie konfiguracji
		\item interfejs pozwala na wczytanie konfiguracji
		\item jednostka może zaatakować wrogą jednostkę
		\item jednostka może zginąć
		\item zaatakowana jednostka sprawdza lokację, z której padł strzał
		\item antyterrorysta może zostać liderem, jeśli ten zginie
	\end{itemize}
\\\hline
\end{tabular}
\caption {Zestaw scenariuszy dla funkcjonalności czwartego etapu\label{scenarios4}}
\end{center}
\end{table} 

Ostatni etap implementacji składał się z mniejszych funkcjonalności, które miały swoje źródła w informacji zwrotnej uzyskanej podczas testów. Antyterroryści podążający za liderem wyposażeni są w detekcję kolizji ze ścianami, co pozwala na ich bezpieczne omijanie. Terroryści natomiast reagują na dźwięk wystrzału, kierując się do jego źródła. Dodatkowo interfejs użytkownika jest wzbogacony o statystyki jednostek, a gra symulacyjna posiada dźwięki odgrywane podczas startu symulacji oraz przy oddawaniu strzałów. Szczegóły zostały przedstawione w tabeli \ref{scenarios5}.

\begin{table}
\begin{center}
\begin{tabular}{|p{1.0\textwidth}|}
\hline
ETAP 5 - Pożądane funkcjonalności\\\hline
	\begin{itemize}
		\setlength\itemsep{0pt}
		\item antyterrorysta, nie będący liderem, może aktywnie omijać ściany
		\item terrorysta reaguje na dźwięk wystrzału i trafienia (w określonym promieniu) podążając do jego źródła
		\item interfejs może wyświetlać statystyki dla jednostek (pozostałe życie, liczba zabić)
		\item aplikacja może odtwarzać dźwięki
	\end{itemize}
\\\hline
\end{tabular}
\caption {Zestaw scenariuszy dla funkcjonalności piątego etapu\label{scenarios5}}
\end{center}
\end{table} 

Prócz scenariuszy, użytecznym elementem specyfikacji był szkic interfejsu użytkownika. Podczas implementacji nanoszone były na niego nieznaczne zmiany. Ostateczna wersja szkicu jest zaprezentowana na rysunku \ref{wireframe}.

\begin{figure}
\begin{center}
	\includegraphics[width=160mm,height=106mm]{images/wireframe}
	\caption{Końcowy szkic interfejsu użytkownika\label{wireframe}}
\end{center}
\end{figure}

\section{Wymagania niefunkcjonalne}
Zestaw wymagań niefunkcjonalnych dla gry symulacyjnej, będącej przedmiotem tej pracy dyplomowej, został przedstawiony w tabeli \ref{nonFunc}. Po zaimplementowaniu aplikacji, została pozytywnie zweryfikowana zgodność z przywołanymi przeglądarkami\footnote{do weryfikacji została użyta usługa http://www.browserstack.com/}.

\begin{table}
\begin{center}
\begin{tabular}{|p{1.0\textwidth}|}
\hline
	\begin{itemize}
		\setlength\itemsep{0pt}		
		\item System operacyjny: Windows, Linux lub MacOS
		\item Przeglądarka internetowa: 
			\begin{itemize}
				\item Chrome w wersji 15.0 lub wyższej
				\item Firefox w wersji 4.0 lub wyższej
				\item Internet Explorer w wersji 9.0 lub wyższej
				\item Safari w wersji 5.1 lub wyższej
			\end{itemize} 
	\end{itemize}
\\\hline
\end{tabular}
\caption {Lista wymagań niefunkcjonalnych\label{nonFunc}}
\end{center}
\end{table} 

\section{Słownik pojęć}\label{dictonary}
Podczas sporządzania specyfikacji gry symulacyjnej, która jest przedmiotem tej pracy dyplomowej, niezbędne było dokładne zdefiniowanie niektórych wykorzystywanych pojęć. Poniżej znajduje się lista pojęć, uporządkowana alfabetycznie.

\begin{description}
	\item[Antyterrorysta] jest to jednostka, która w grze symulacyjnej oznaczona jest kolorem niebieskim. Celem antyterrorysty jest eliminacja wszystkich terrorystów
	\item[Antyterrorysta lider] jest to antyterrorysta, który prowadzi oddział antyterrorystyczny. Reszta antyterrorystów podąża za liderem. Liderem jest wybierana pierwsza żyjąca jednostka na liście antyterrorystów
	\item[Interfejs] jest to część aplikacji, która służy do przygotowania konfiguracji, sterowania symulacją oraz prezentacji logów i statystyk jednostek
	\item[Jednostka] jest to obiekt ruchomy, wykazujący pewne działanie taktyczne. W grze symulacyjnej jednostkami są terroryści i antyterroryści
	\item[Konfiguracja] są to dane o położeniu ścian, punktu startowego / końcowego antyterrorystów oraz punktów kluczowych. Konfiguracja może być zapisana, wczytana lub usunięta z poziomu interfejsu
	\item[Obiekt ruchomy] jest nim każda jednostka oraz każdy pocisk
	\item[Punkt kluczowy] jest to punkt należący do uporządkowanego zbioru, na podstawie którego budowane są ścieżki dla antyterrorystów. Wokół punktów kluczowych tworzeni są terroryści na początku rozgrywki
	\item[Punkt startowy / końcowy] jest to punkt, w którym są tworzeni i do którego wracają antyterroryści po przejściu przez wszystkie punkty kluczowe
	\item[Scena] jest to część interfejsu ukazująca mapę lokacji oraz ruchome obiekty
	\item[Statystyki] jest to część interfejsu ukazująca aktualny stan punktów życia oraz ilość zabić dla poszczególnych jednostek.
	\item[Symulacja] jest to stan gry, w którym na scenie znajdują się jakiekolwiek jednostki	
	\item[Terrorysta] jest to jednostka, która w grze symulacyjnej oznaczona jest kolorem czerwonym. Celem terrorysty jest obrona terytorium przed antyterrorystami
	\item[Warstwa sceny] jest to część sceny, do której aplikacja może przypisać obiekty w~celu późniejszego renderowania.
	\item[Zamknięcie konfliktu] jest to sytuacja, w której nie żyją wszyscy antyterroryści lub nie żyją wszyscy terroryści.
\end{description}




% Rozdział 3
\chapter{Projekt symulatora operacji antyterrorystycznych}
\section{Model obiektowy}
Wykorzystywany podczas implementacji Javascript, jako skryptowy język programowania, nie jest językiem ściśle obiektowym (jak np. Java), lecz mimo wszystko umożliwia on pisanie aplikacji technikami programowania obiektowego. Do tego celu służy m. in. prototypowanie, które pozwala także na dziedziczenie przygotowywanych klas.

W grze symulacyjnej, będącej przedmiotem tej pracy dyplomowej, możemy wyróżnić trzy obiekty, które wywodzą się bezpośrednio z klasy obiektu javascriptowego oraz dziesięć klas, które dziedziczą atrybuty i metody z różnych klas kształtów, zawartych w bibliotece Kinetic.js. Nazwy metod, które są postrzegane jako prywatne dla danej klasy, rozpoczynają się od znaku podkreślenia "\_"\footnote{w Javascript'cie nie ma dedykowanego mechanizmu rozróżniania metod prywatnych i publicznych}. W prezentowanych tabelach zostały pominięte atrybuty i metody odziedziczone z innych klas. Definicje klas wchodzących w skład biblioteki Kinetic.js można znaleźć na stronie projektu\cite{kineticPage}.

Obiekt \textbf{Game} zawiera informacje dotyczące świata gry, tj. jego wymiarów, aktualnej konfiguracji oraz funkcjonujących jednostek. Stanowi on interfejs dla obiektów innych klas, przez który mogą one dostrzegać zmiany zachodzące w świecie gry. Metody zawarte w obiekcie Game pozwalają ma kontrolowanie symulacji. Opis atrybutów oraz zaimplementowanych metod znajduje się w tabeli \ref{objectsGame}.

\begin{table}
\begin{center}
\begin{tabular}{|p{0.28\textwidth}|p{0.72\textwidth}|}
\hline
\textbf{Game} & Opis\\\hline		
	width & szerokość sceny wyrażona w pikselach\\
	height & wysokość sceny wyrażona w pikselach\\
	mapDensity & wymiar jednego kafelka na mapie, na potrzeby reprezentacji grafowej\\
	uiState & aktualny stan interfejsu graficznego, określa która strona konfiguracji jest aktualnie otwarta\\
	stage & obiekt sceny zawierający poszczególne warstwy\\
	map & obiekt mapy zawierający część informacji o konfiguracji\\ 
	entities & warstwa sceny zawierająca istniejące w symulacji jednostki (terroryści i antyterroryści)\\
	configObjects & warstwa sceny zawierająca obiekty wspomagające konfigurację (szkic punktu kluczowego, szkic punktu startowego itp.)\\
	mapObjects & warstwa sceny zawierająca obiekty należące do mapy (ściany, punkty kluczowe itp.)\\
	paused & zmienna logiczna informująca o włączeniu/wyłączeniu pauzy\\
	antiterroristsCount & liczba antyterrorystów wynikająca z konfiguracji\\
	terroristsCount & liczba terrorystów wynikająca z konfiguracji\\
	keypointIndex & numer aktualnie realizowanego punktu kluczowego przez antyterrorystów
\\\hline
	init & inicjalizuje aplikację tworząc scenę oraz warstwy\\
	initMap & tworzy obiekt mapy\\
	togglePause & przyłącza stan pauzy\\
	startGame & rozpoczyna symulację\\ 
	endGame & kończy symulację\\
	getEntities & zwraca listę wszystkich jednostek istniejących w bieżącej symulacji\\
	getAliveTerrorists & zwraca listę niezabitych terrorystów w bieżącej symulacji\\ 
	getAliveAntiterrorists & zwraca listę niezabitych antyterrorystów w bieżącej symulacji\\
	checkAliveEntities & sprawdza stan jednostek, a w razie zaistnienia zamknięcia konfliktu, tworzy odpowiedni wpis w logach\\
	getNodeByPosition & zwraca węzeł w grafie na podstawie zadanych współrzędnych\\
	\_spawnTerrorists & tworzy obiekty terrorystów podczas startu symulacji\\ 
	\_spawnAntiterrorists & tworzy obiekty antyterrorystów podczas startu symulacji
\\\hline
\end{tabular}
\caption {Obiekt gry - Game\label{objectsGame}}
\end{center}
\end{table} 

Obiekt \textbf{GameControl} zawiera przede wszystkim metody, które wiążą interfejs z~obiektem Game. Są tutaj zdefiniowane wszystkie metody wywoływane poprzez kliknięcia użytkownika w przyciski znajdujące się na interfejsie. Opis atrybutów oraz zaimplementowanych metod znajduje się w tabeli \ref{objectsGameControl}. 

\begin{table}
\begin{center}
\begin{tabular}{|p{0.28\textwidth}|p{0.72\textwidth}|}
\hline
\textbf{GameControl} & Opis\\\hline		
	storagePrefix & stała zawierająca informację o prefiksie dla nazw zapisywanych konfiguracji\\
	simStartTime & czas rozpoczęcia symulacji wyrażony w milisekundach\\
	winMessage & wiadomość o ew. zwycięstwie jednej ze stron konfliktu
\\\hline
	init & inicjalizuje interfejs, tworzy powiązania z obiektem Game\\
	log & umieszcza wpis o zadanym tekście w logach\\
	setWinMessage & tworzy wiadomość dotyczącą zwycięstwa jednej ze stron konfliktu\\
	configs & zwraca listę wcześniej zapisanych konfiguracji\\
	loadConfig & wczytuje wybraną konfigurację\\
	saveConfig & zapisuje bieżącą konfigurację\\
	removeConfig & usuwa wybraną konfigurację\\
	startSim & powiązana z przyciskiem rozpoczynającym symulację \\
	pauseSim & powiązana z przyciskiem wstrzymującym symulację\\ 
	stopSim & powiązana z przyciskiem zatrzymującym symulację\\ 
	removeLastWall & powiązana z przyciskiem usuwającym ostatnio utworzoną ścianę\\
	clearWalls & powiązana z przyciskiem usuwającym wszystkie ściany\\
	removeSpawnZone & powiązana z przyciskiem usuwającym punkt startowy / końcowy dla antyterrorystów\\
	removeLastKeypoint & powiązana z przyciskiem usuwającym ostatnio utworzony punkt kluczowy\\
	clearKeypoints & powiązana z przyciskiem usuwającym wszystkie punkty kluczowe\\
	nextConfig & otwiera następną stronę konfiguracji\\
	previousConfig & otwiera poprzednią stronę konfiguracji\\
	changeUiState & otwiera zadaną stronę konfiguracji\\
	clearEntitiesList & usuwa dane jednostek ze statystyk\\
	createEntitiesList & dodaje dane jednostek do statystyk\\
	updateStat & aktualizuje statystyki dla danej jednostki\\
	\_updateCursor & zmienia styl kursora myszy nad sceną\\
	\_updateNumberData & zmienia dane liczbowe o jednostkach w obiekcie Game\\
	\_updateConfigStatus & zwraca informację o ew. niekompletnej konfiguracji 
\\\hline
\end{tabular}
\caption {Obiekt kontroli gry - GameControl\label{objectsGameControl}}
\end{center}
\end{table} 

Obiekt \textbf{Sounds} zawiera definicję ścieżek do plików dźwiękowych wykorzystywanych w grze oraz metody pozwalające je odtwarzać i zatrzymywać. W grze symulacyjnej zdefiniowane są trzy dźwięki: dźwięk rozpoczynający symulację, wystrzały terrorystów oraz wystrzały antyterrorystów. Opis atrybutów oraz zaimplementowanych metod znajduje się w tabeli \ref{objectsSounds}. 

\begin{table}
\begin{center}
\begin{tabular}{|p{0.28\textwidth}|p{0.72\textwidth}|}
\hline
\textbf{Sounds} & Opis\\\hline		
	list & tablica asocjacyjna zawierająca ścieżki do plików dźwiękowych\\
	instances & tablica asocjacyjna zwierająca instancje odgrywanych plików dźwiękowych
\\\hline
	init & tworzy powiązanie z obiektem Game\\
	play & rozpoczyna odtwarzanie danego dźwięku\\
	stop & zatrzymuje odtwarzanie danego dźwięku
\\\hline
\end{tabular}
\caption {Obiekt dźwięków gry - Sounds\label{objectsSounds}}
\end{center}
\end{table} 

Obiekt klasy \textbf{Game.Map} rozszerza klasę Kinetic.Rect. Zawiera on referencje do elementów konfiguracji symulacji - ściany, punkty kluczowe. Ponadto obiekt mapy zawiera graf reprezentujący stan poszczególnych pól na mapie (zajęte lub niezajęte), co jest potrzebne podczas generowania ścieżek dla poruszających się jednostek. Obiekt mapy posiada także metody pozwalające na serializowanie konfiguracji do formatu JSON\footnote{JavaScript Object Notation - tekstowy format wymiany danych, alternatywa dla XML} oraz do importu konfiguracji dostarczonej w takim formacie. Obsługa zdarzeń nad sceną, których źródłem jest urządzenie wskazujące, jest również zaimplementowana na obiekcie mapy. Opis atrybutów oraz zaimplementowanych metod znajduje się w tabeli \ref{objectsGameMap}.  

\begin{table}
\begin{center}
\begin{tabular}{|p{0.28\textwidth}|p{0.72\textwidth}|}
\hline
\textbf{Game.Map} & Opis\\\hline		
	newWall & obiekt szkicu tworzonej ściany\\
	graph & obiekt grafu, niezbędnego do wytyczania ścieżek\\
	zone & obiekt punktu startowego / końcowego antyterrorystów\\
	zoneDraft & obiekt szkicu punktu startowego / końcowego antyterrorystów\\
	newKeypoint & obiekt szkicu punktu kluczowego\\
	walls & tablica zawierająca istniejące ściany\\
	keypoints & tablica zawierająca istniejące punkty kluczowe
\\\hline
	init & inicjalizuje obiekt mapy\\ 
	removeLastWall & usuwa ostatnio utworzoną ścianę\\
	clearWalls & usuwa wszystkie ściany\\
	removeLastKeypoint & usuwa ostatnio utworzony punkt kluczowy\\
	clearKeypoints & usuwa wszystkie punkty kluczowe\\
	removeZone & usuwa punkt startowy / końcowy antyterrorystów\\
	serializeConfig & serializuje bieżącą konfigurację\\
	importConfig & deserializuje dostarczoną konfigurację i tworzy nowe obiekty na jej podstawie\\
	\_bindEvents & inicjalizuje obsługę zdarzeń nad sceną\\
	\_buildGraph & inicjalizuje graf\\
	\_buildGrid & buduje siatkę nad sceną\\
	\_initWall & inicjalizuje nową ścianę\\
	\_updateWall & uaktualnia obiekt szkicu ściany\\
	\_addWall & dodaje utworzoną ścianę do listy ścian\\
	\_updateWallOnGraph & oznacza węzły grafu, które pokrywa zadana ściana\\
	\_buildZoneDraft & inicjalizuje punkt startowy / końcowy antyterrorystów\\
	\_buildKeypoint & inicjalizuje punkt kluczowy\\
	\_showDraftZone & pokazuje szkic punktu startowego / końcowego antyterrorystów gdy ukryty\\
	\_hideDraftZone & ukrywa szkic punktu startowego / końcowego antyterrorystów gdy widoczny\\
	\_setZone & ustanawia punkt startowy / końcowy antyterrorystów\\ 
	\_updateDraftZone & uaktualnia położenie szkicu punktu startowego / końcowego antyterrorystów \\
	\_addKeypoint & dodaje utworzony punkt kluczowy do listy punktów kluczowych\\
	\_showNewKeypoint & pokazuje szkic punktu kluczowego gdy ukryty\\ 
	\_hideNewKeypoint & ukrywa szkic punktu kluczowego gdy widoczny\\
	\_updateNewKeypoint & uaktualnia położenie szkicu punktu kluczowego 
\\\hline
\end{tabular}
\caption {Klasa mapy - Game.Map\label{objectsGameMap}}
\end{center}
\end{table} 


Klasa \textbf{Game.Line} rozszerza klasę Kinetic.Line. Sama stanowi podstawę dla klas Game.GridLine oraz Game.Wall. Klasa Game.line zawiera metody sprawdzające przecięcia linii z inną linią oraz linii z okręgiem. Metody te są wykorzystywane w wielu metodach należących do obiektów ruchomych (klasa Game.Entity). Opis atrybutów oraz zaimplementowanych metod znajduje się w tabeli \ref{objectsGameLine}.   

\begin{table}
\begin{center}
\begin{tabular}{|p{0.45\textwidth}|p{0.55\textwidth}|}
\hline
\textbf{Game.Line} & Opis\\\hline		
	 & \emph{klasa ta zawiera wyłącznie atrybuty dziedziczone}
\\\hline
	init & inicjalizuje obiekt linii\\
	getStartPoint & zwraca współrzędne punktu początkowego linii\\
	getEndPoint & zwraca współrzędne punktu końcowego linii\\
	getVecStartPoint & zwraca punkt początkowy linii w postaci wektora\\
	getVecEndPoint & zwraca punkt końcowy linii w postaci wektora\\ 
	setStartPoint & ustawia punkt początkowy linii\\
	setEndPoint & ustawia punkt końcowy linii\\
	getIntersectionPointWithLine & zwraca współrzędne punktu przecięcia dwóch linii\\
	getVecIntersectionPointWithSphere & zwraca punkt przecięcia linii i okręgu w postaci wektora\\
	getVecIntersectionPoint & zwraca punkt przecięcia dwóch linii w postaci wektora\\
	getNormals & zwraca wektory normalne dla danej linii\\
	\_getClosestPointOnLine & zwraca najbliższy punkt na linii do zadanego punktu 
\\\hline
\end{tabular}
\caption {Klasa linii - Game.Line\label{objectsGameLine}}
\end{center}
\end{table} 

Klasa \textbf{Game.Wall} rozszerza klasę Game.Line. Instancje tej klasy reprezentują ściany w grze symulacyjnej. Klasa zawiera dodatkowo metodę sprawdzającą poprawność budowanej ściany, która musi być linią poziomą lub pionową, a nie może być linią skośną. Ściany w grze stanowią dla jednostek jedyną przeszkodę, którą jednostki muszą omijać. Utworzone ściany mają swoje odzwierciedlenie na grafie w postaci niedostępnych dla jednostek węzłów. Opis atrybutów oraz zaimplementowanych metod znajduje się w tabeli \ref{objectsGameWall}. 

\begin{table}
\begin{center}
\begin{tabular}{|p{0.28\textwidth}|p{0.72\textwidth}|}
\hline
\textbf{Game.Wall} & Opis\\\hline		
	valid & zawiera informację czy ściana jest poprawna
\\\hline
	init & inicjalizuje obiekt ściany\\
	setEndPoint & nadpisana metoda klasy Game.Line, dodatkowo wywołuje metodę \_validate\\
	isVertical & zwraca informację czy linia jest pionowa\\
	isHorizontal & zwraca informację czy linia jest pozioma\\
	\_validate & sprawdza poprawność zbudowanej linii
\\\hline
\end{tabular}
\caption {Klasa ściany - Game.Line\label{objectsGameWall}}
\end{center}
\end{table} 

Klasa \textbf{Game.Keypoint} rozszerza klasę Kinetic.Text. Instancje tej klasy reprezentują punkty kluczowe w grze symulacyjnej. Klasa zawiera metodę sprawdzającą poprawność tworzonego punktu kluczowego, który nie może leżeć w miejscu gdzie jest ściana. Podczas rozpoczynania symulacji, terroryści są tworzeni w losowo wybranych punktach kluczowych. Punkty te jednocześnie wytyczają trasę jaką muszą pokonać antyterroryści podczas przeprowadzanego szturmu. Antyterrorysta lider, po dotarciu do danego punktu kluczowego, wytycza bezkolizyjną ścieżkę do kolejnego punktu. Opis atrybutów oraz zaimplementowanych metod znajduje się w tabeli \ref{objectsGameKeypoint}. 

\begin{table}
\begin{center}
\begin{tabular}{|p{0.28\textwidth}|p{0.72\textwidth}|}
\hline
\textbf{Game.Keypoint} & Opis\\\hline		
	valid & zawiera informację czy punkt kluczowy jest poprawny
\\\hline
	init & inicjalizuje obiekt punktu kluczowego\\
	updatePosition & uaktualnia położenie punktu kluczowego\\
	\_validate & sprawdza poprawność tworzonego punktu kluczowego
\\\hline
\end{tabular}
\caption {Klasa punktu kluczowego - Game.Keypoint\label{objectsGameKeypoint}}
\end{center}
\end{table} 

Klasa \textbf{Game.Entity} rozszerza klasę Kinetic.Image. Sama stanowi podstawę dla klas Game.Terrorist, Game.Antiterrorist oraz Game.Bullet. Klasa ta posiada metody pozwalające wyliczać wektor prędkości dla algorytmów poruszania się oraz sprawdzać kolizję z innymi obiektami. Część atrybutów i metod jest tutaj odpowiedzialna za realizację wspólnych dla terrorystów i antyterrorystów taktyk. Dzięki implementacji metody \emph{update}, położenie obiektów na scenie jest stale aktualizowane. Opis atrybutów znajduje się w tabeli \ref{objectsGameEntityAttrs}, natomiast opis zaimplementowanych metod znajduje się w tabeli \ref{objectsGameEntityFuncs}. 

\begin{table}[p]
\begin{center}
\begin{tabular}{|p{0.28\textwidth}|p{0.72\textwidth}|}
\hline
\textbf{Game.Entity} & Opis atrybutów\\\hline		
    imageSrc & ścieżka do pliku z reprezentacją graficzną\\
    maxSpeed & maksymalna prędkość obiektu\\
    velX & współrzędna X wektora prędkości\\
    velY & współrzędna Y wektora prędkości\\
    tarX & współrzędna X celu\\
    tarY & współrzędna Y celu\\
    rayLine & obiekt linii, która sprawdza możliwość wystąpienia kolizji\\
    groupIndex & numer obiektu w danej stronie konfliktu\\
    isAlive & zawiera informację czy obiekt żyje\\
    dieAlpha & stopień przezroczystości, jaka jest stosowana, gdy obiekt ginie\\
    speed & aktualna prędkość\\
    avoidDistance & dystans, jaki obiekt powinien zachowywać od kolidujących obiektów\\
    lookAhead & długość linii, sprawdzającej możliwość wystąpienia kolizji\\
    arrivePrecision & dokładność, z jaką się określa czy obiekt dotarł do celu\\
    targetEntity & obiekt ruchomy, który jest aktualnym celem\\
    watchedEntity & obiekt ruchomy, który jest obserwowany, ale nie atakowany\\
    healthPoints & liczba punktów życia\\
    healthPointsMax & liczba punktów życia, jakie obiekt posiada po utworzeniu\\
    collisionRadius & promień okręgu wytyczającego strefę kolizyjną obiektu\\
    kills & liczba zabić dokonanych przez obiekt\\
    nodeIndex & indeks aktualnie odwiedzanego węzła na ścieżce\\
    path & ścieżka reprezentowana przez tablicę węzłów\\
    currentState & nazwa aktualnego stanu obiektu\\
    checkDirectionTimeMax & maksymalny czas, jaki może zostać poświęcony na sprawdzenie kierunku\\
    checkDirectionTime & aktualny czas, jaki pozostał na sprawdzenie kierunku\\
    sightDistance & zasięg wzroku obiektu\\
    name & nazwa typu obiektu\\
    enemyName & nazwa wrogiego typu obiektu
\\\hline
\end{tabular}
\caption {Atrybuty klasy obiektu ruchomego - Game.Entity\label{objectsGameEntityAttrs}}
\end{center}
\end{table} 

\begin{table}
\begin{center}
\begin{tabular}{|p{0.34\textwidth}|p{0.66\textwidth}|}
\hline
\textbf{Game.Entity} & Opis metod\\\hline		
	init & inicjalizuje obiekt ruchomy\\
	setTarget & ustawia cel wg zadanych współrzędnych\\
	setTargetEntity & ustawia inny obiekt jako swój cel\\
	currentTargetEntity & zwraca aktualny obiekt będący celem\\
	unsetTargetEntity & usuwa przypisanie celu, który jest obiektem\\
	updateTargetEntity & uaktualnia wiedzę o położeniu celu\\
	setVelocity & ustawia wektor prędkości\\
	hasVelocity & zwraca informację czy obiekt się porusza\\
	getVecPosition & zwraca aktualną pozycję w postaci wektora\\
	getVecVelocity & zwraca aktualną prędkość w postaci wektora\\
	getVecTarget & zwraca aktualną pozycję celu w postaci wektora\\
	update & przesuwa obiekt o zadany wektor prędkości oraz aktualizuje orientację\\
	changeState & zmienia stan obiektu na zadany\\
	die & uśmierca obiekt\\
	setRandomPositionInCircle & wybiera losową pozycję obiektu wokół określonego okręgu\\
	isInCollision & sprawdza czy obiekt nie koliduje ze ścianą lub inną istniejącą jednostką\\
	seek & aktualizuje wektor prędkości w kierunku do celu\\
	flee & aktualizuje wektor prędkości w kierunku przeciwnym do celu\\
	stop & zatrzymuje obiekt\\
	arrived & sprawdza czy obiekt dotarł do celu\\
	checkForCollision & sprawdza czy obiekt nie koliduje ze ścianą\\
	closestSeenOpponent & zwraca najbliższego przeciwnika w zasięgu wzroku\\
	takeDamage & zadaje rany obiektowi poprzez odbiór punktów życia\\
	watchForEnemy & metoda pozwalająca na obserwowanie przeciwnika i po określonym czasie przejście do ataku\\
	attack & atakowanie przeciwnika poprzez oddawanie strzałów co określony interwał czasowy\\
	calculatePath & wyliczanie ścieżki do zadanego celu\\
	checkDirection & metoda pozwalająca na przejście po ścieżce do wcześniej wytyczonego celu\\
	setCheckDirection & metoda określająca cel, do którego należy dotrzeć wg wytyczonej ścieżki\\
	\_logDeath & tworzenie wpisu w logach o śmierci obiektu\\
	\_updateCollisionRay & uaktualnia położenie linii, która sprawdza możliwość wystąpienia kolizji\\
	\_calculateVelocity & wyliczanie wektora prędkości
\\\hline
\end{tabular}
\caption {Metody klasy obiektu ruchomego - Game.Entity\label{objectsGameEntityFuncs}}
\end{center}
\end{table} 

Klasa \textbf{Game.Antiterrorist} rozszerza klasę Game.Entity. Instancje tej klasy reprezentują antyterrorystów w grze symulacyjnej. Najważniejszą metodą zawartą w~tej klasie jest \emph{think}. Decyduje ona o działaniach antyterrorysty poprzez umożliwienie zmiany stanów. Opis atrybutów oraz zaimplementowanych metod znajduje się w tabeli \ref{objectsGameAntiterrorist}.

Klasa \textbf{Game.Terrorist} rozszerza klasę Game.Entity. Instancje tej klasy reprezentują terrorystów w grze symulacyjnej. Podobnie jak w klasie Game.Antiterrorist, kluczową rolę w tej klasie odgrywa metoda \emph{think}. Opis atrybutów oraz zaimplementowanych metod znajduje się w tabeli \ref{objectsGameTerrorist}. Dodatkowo w rozdziale \ref{tacticts} jest przedstawiony diagram przejść międzystanowych, który w ilustruje taktykę charakterystyczną dla działań antyterrorystów i terrorystów. 

\begin{table}
\begin{center}
\begin{tabular}{|p{0.28\textwidth}|p{0.72\textwidth}|}
\hline
\textbf{Game.Antiterrorist} & Opis\\\hline		
	reactionTimeMax & czas reakcji przejścia  z obserwowania do ataku\\
	reactionTime & aktualny czas pozostały do przejścia do ataku\\
	shootInterval & czas pomiędzy kolejnymi wystrzałami\\
	shootTime & aktualny czas pozostały do wystrzału\\
	followDistance & odległość, w jakiej antyterrorysta podąża za poprzednikiem\\
	isLeader & zawiera informację, czy antyterrorysta jest liderem\\
	keypointIndex & numer aktualnie realizowanego punktu kluczowego przez lidera antyterrorystów
\\\hline
	think & metoda odpowiedzialna za realizację taktyk\\
	followEntity & podążanie w linii za poprzednikiem\\ 
	followPath & podążanie do następnego punktu kluczowego po ścieżce\\
	followExtraction & podążanie do punktu startowego / końcowego antyterrorystów\\
	avoid & omijanie napotkanej ściany poprzez wytyczenie ścieżki\\
	changeToDefaultState & zmiana stanu do domyślnego (dla antyterrorystów jest to \emph{followEntity})\\
	\_reactOnDamage & reakcja na postrzał (dla antyterrorystów jest to sprawdzenie kierunku z którego padł strzał)
\\\hline
\end{tabular}
\caption {Klasa antyterrorysty - Game.Antiterrorist\label{objectsGameAntiterrorist}}
\end{center}
\end{table} 


\begin{table}
\begin{center}
\begin{tabular}{|p{0.28\textwidth}|p{0.72\textwidth}|}
\hline
\textbf{Game.Terrorist} & Opis\\\hline		
	reactionTimeMax & czas reakcji przejścia  z obserwowania do ataku\\
	reactionTime & aktualny czas pozostały do przejścia do ataku\\
	shootInterval & czas pomiędzy kolejnymi wystrzałami\\
	shootTime & aktualny czas pozostały do wystrzału\\
	wanderCircleDistance & odległość środka okręgu wyznaczającego kurs wędrówki od terrorysty\\
	wanderRadius & promień okręgu wyznaczającego kurs wędrówki\\
	wanderRate & zakres wahania kierunku wędrówki\\
	wanderOrientation & orientacja kierunku wędrówki\\
	standingProbability & prawdopodobieństwo przejścia do stanu postoju\\
	standingTimeMax & maksymalny czas, jaki może trwać pojedynczy postój\\
	standingTime & czas pozostały do zakończenia aktualnego postoju
\\\hline
	think & metoda odpowiedzialna za realizację taktyk\\
	stand & metoda odpowiedzialna za sprawdzanie czy postój ma nadal trwać\\
	wander & poruszanie się zgodnie z kierunkiem wędrówki\\
	avoid & omijanie napotkanej ściany poprzez skierowanie terrorysty do punktu wyznaczanego przez normalną ściany\\
	changeToDefaultState & zmiana stanu do domyślnego (dla terrorystów jest to \emph{wander})\\
	\_wantToStand & sprawdzenie czy terrorysta na wykonać postój\\
	\_reactOnDamage & reakcja na postrzał (dla terrorystów jest to sprawdzenie kierunku z którego padł strzał)
\\\hline
\end{tabular}
\caption {Klasa terrorysty - Game.Terrorist\label{objectsGameTerrorist}}
\end{center}
\end{table} 

Klasa \textbf{Game.Bullet} rozszerza klasę Game.Entity. Instancje tej klasy reprezentują pociski w grze symulacyjnej. Podczas przemieszczania się pocisku, sprawdzane jest czy nie trafił on w ścianę lub jednostkę. Symulowany odgłos wystrzału oraz trafienia może przyciągać uwagę terrorystów znajdujących się w określonej odległości od pocisku. Im dłużej pocisk się porusza, tym mniejszą posiada energię, która decyduje o ew. ranach zadanych jednostce. Opis atrybutów oraz zaimplementowanych metod znajduje się w tabeli \ref{objectsGameBullet}. 

\begin{table}
\begin{center}
\begin{tabular}{|p{0.30\textwidth}|p{0.70\textwidth}|}
\hline
\textbf{Game.Bullet} & Opis\\\hline		
	shooter & referencja do obiektu jednostki,  która wystrzeliła pocisk\\
	energy & energia, jaką aktualnie posiada pocisk\\
	bulletRange & zasięg pocisku\\	
	attentionRange & promień w jakim wystrzelony pocisk jest słyszalny
\\\hline
	move & metoda odpowiedzialna za ruch pocisku i sprawdzenie ew. trafienia\\
	\_drawTerroristsAttention & zmiana stanu terrorystów będących w zasięgu słyszalności wystrzału lub trafienia\\
	\_playSound & odtwarzanie dźwięku wystrzału
\\\hline
\end{tabular}
\caption {Klasa pocisku - Game.Bullet\label{objectsGameBullet}}
\end{center}
\end{table} 
\clearpage
Poza wymienionymi klasami, w grze symulacyjnej zdefiniowane są jeszcze klasy, które wyłącznie nadpisują konfigurację wyświetlania kształtu. Należą do nich:
\begin{itemize}
	\item Game.Zone - dziedziczy z klasy Kinetic.Circle. Instancje tej klasy reprezentują punkt startowy / początkowy antyterrorystów
	\item Game.GridLine - dziedziczy z klasy Kinetic.Line. Instancje tej klasy reprezentują siatkę narzuconą na scenę
\end{itemize}

\section{Sztuczna inteligencja - taktyki}
W grze symulacyjnej, będącej przedmiotem niniejszej pracy dyplomowej, terroryści oraz antyterroryści posiadają sztuczną inteligencję\footnote{charakterystyka modelu SI została przedstawiona w rozdziale \ref{aiModelInfo}}, która pozwala im na podejmowanie decyzji oraz poruszanie się. Interfejsem, z którego jednostki uczestniczące w symulacji czerpią wiedzę o świecie gry, jest obiekt Game. 

Terroryści nie posiadają grupowej strategii działania, kierują się wyłącznie indywidualnie podejmowanymi decyzjami. Natomiast antyterroryści posiadają strategię, która nakazuje im posiadanie lidera przez cały czas trwania symulacji. Lider antyterrorystów jest jednostką, za którą w szyku poruszają się pozostali antyterroryści. Jeżeli lider zginie, to natychmiastowo wybierany jest nowy lider, a działania grupy są kontynuowane.
  
Antyterrorysta posiada skończony zbiór stanów (rysunek \ref{atTacticImage}), które odzwierciedlają podjęte przez niego decyzje i definiują jego działania. Po zainicjalizowaniu obiektu antyterrorysty przechodzi on do stanu \emph{follow entity}, który pozwala mu podążać za swoimi poprzednikami. Jest to domyślny stan, do którego antyterrorysta może wrócić ze stanów, do których przeszedł na podstawie zdarzenia. Jeżeli antyterrorysta jest liderem, to następuje natychmiastowe przejście do stanu \emph{follow path}, które definiuje konieczność poruszania się po wyznaczonej ścieżce do następnego punktu kluczowego. Wraz z dotarciem do danego punktu kluczowego, wyznaczana jest ścieżka do następnego punktu kluczowego. Jeżeli antyterrorysta lider dotarł do ostatniego punktu kluczowego, to zmienia on swój stan na \emph{follow extraction}, który nakazuje jednostce poruszanie się po ścieżce do punktu startowego / końcowego antyterrorystów. Po dotarciu do tego punktu antyterrorysta zatrzymuje się i przechodzi do stanu bezczynności - \emph{idle}.

\begin{figure}
\begin{center}
	\includegraphics[width=160mm,height=167mm]{images/atTactic}
	\caption{Diagram przejść międzystanowych antyterrorysty\label{atTacticImage}}
\end{center}
\end{figure}

Każdy antyterrorysta może zmienić swój stan na podstawie zaistniałego zdarzenia. Postrzelenie antyterrorysty lub natrafienie przez niego na ścianę, wiąże się z natychmiastowym wywołaniem stanu \emph{check direction}. Stan ten definiuje poruszanie się jednostki do zadanego punktu, z wykorzystaniem wyliczonej ścieżki bezkolizyjnej. Antyterrorysta opuszcza ten stan po dotarciu do celu lub po upłynięciu limitu czasowego na wykonanie tej czynności. Zdarzenie polegające na zauważeniu przeciwnika, wywołuje stan \emph{attack}. Stan ten pozwala antyterroryście na oddawanie strzałów w kierunku zauważonego terrorysty, jeżeli na linii strzału nie znajduje się żaden antyterrorysta. Wyjście z tego stanu następuje po zabiciu terrorysty lub po straceniu celu z pola widzenia.

Terrorysta również posiada skończony zbiór stanów (rysunek \ref{terTacticImage}). Po zainicjalizowaniu obiektu terrorysty przechodzi on do stanu \emph{wander}, który pozwala mu na wędrowanie po świecie gry. Podczas wędrowania terrorysta może podjąć decyzję o wykonaniu postoju, co wiąże się z przejściem do stanu \emph{stand}. Stan ten zatrzymuje jednostkę oraz odlicza czas pozostały do końca postoju, a po jego upłynięciu zmienia stan terrorysty ponownie na \emph{wander}.

\begin{figure}
\begin{center}
	\includegraphics[width=160mm,height=139mm]{images/terTactic}
	\caption{Diagram przejść międzystanowych terrorysty\label{terTacticImage}}
\end{center}
\end{figure}

Każdy terrorysta także może zmienić swój stan na podstawie zaistniałego zdarzenia. Postrzelenie terrorysty lub usłyszenie przez niego odgłosu wystrzału skutkuje wywołaniem stanu \emph{check direction}, który definiuje identyczne zachowanie i warunek wyjścia ze stanu, jak w przypadku antyterrorysty. Podobnie jest ze zdarzeniem polegającym na zauważeniu przeciwnika. Wywołuje ono stan \emph{attack}, który nakazuje terroryście strzelać do wrogiej jednostki. Warunek wyjścia z tego stanu jest taki sam, jak u antyterrorysty.

Zdarzeniem wspólnym dla antyterrorystów oraz dla terrorystów jest otrzymanie śmiertelnego trafienia. W tym przypadku natychmiastowo zmieniany jest stan jednostki na bezczynność - \emph{idle}. Jednostka pozostająca w stanie bezczynności nie reaguje na zdarzenia, co skutkuje brakiem możliwości zmiany swego stanu.

\section{Opis algorytmów}
W tym rozdziale zostaną przedstawione wybrane algorytmy, jakie są wykorzystywane w grze symulacyjnej, będącej przedmiotem tej pracy dyplomowej. Słowny opis jest uzupełniony pseudokodami lub implementacją w języku Javascript.

\subsection{Myślenie i poruszanie się jednostek}
Antyterroryści i terroryści w grze symulacyjnej posiadają zaimplementowaną metodę \emph{think} pozwalającą na realizację działań zapisanych w konkretnych stanach (listing \ref{thinkAt} oraz \ref{thinkTer}). Prócz wywołania metod odpowiadającym konkretnym stanom, wywoływane są także metody \emph{watchForEnemy} oraz \emph{checkForCollision}. Ta pierwsza pozwala jednostce na obserwowanie otoczenia w poszukiwaniu przeciwników, natomiast druga na bezpieczne omijanie ścian. Metody odpowiadające za realizację poszczególnych stanów podejmują najczęściej decyzję, gdzie leży następny cel jednostki.

\begin{table}
\begin{center}
\begin{lstlisting}
    think: function(){
        this.watchForEnemy();
        switch(this.currentState) {
            case 'idle': break;
            case 'init': this.setup(); break;
            case 'follow entity': this.followEntity(); break;
            case 'follow path': this.followPath(); break;
            case 'follow extraction': this.followExtraction(); break;
            case 'check direction': this.checkDirection(); break;
            case 'attack': this.attack(); break;
            default: this.changeToDefaultState(); break;
        }
        if (this.avoiding) this.wanderOrientation = this.getRotation();
        this.avoiding = this.checkForCollision();
    }
 \end{lstlisting}
\caption {Metoda think w klasie Game.Antiterrorist}
\label{thinkAt}
\end{center}
\end{table}

\begin{table}
\begin{center}
\begin{lstlisting}
    think: function(){
        this.watchForEnemy();
        switch(this.currentState) {
            case 'idle': break;
            case 'init': this.setup(); break;
            case 'stand': this.stand(); break;
            case 'wander': this.wander(); break;
            case 'check direction': this.checkDirection(); break;
            case 'attack': this.attack(); break;
            default: this.changeToDefaultState(); break;
        }
        if (this.avoiding) this.wanderOrientation = this.getRotation();
        this.avoiding = this.checkForCollision();
    }
 \end{lstlisting}
\caption {Metoda think w klasie Game.Terrorist}
\label{thinkTer}
\end{center}
\end{table}

Gdy cel jest już zdefiniowany, to wykonywane są algorytmy niższego poziomu, odpowiedzialne za obliczenie wektora prędkości oraz prędkości, z jaką ma poruszać się jednostka. Takimi algorytmami są \emph{seek} (ruch w kierunku celu) oraz \emph{flee} (ruch w przeciwnym kierunku do celu). Następnie otrzymane dane są wykorzystywane przez funkcję \emph{update}, która przesuwa jednostkę na ekranie i nadaje jej odpowiednią orientację.

\subsection{Wyznaczanie ścieżki - A*}
Wyznaczanie ścieżek bezkolizyjnych (uwzględniających położenie ścian na mapie) jest realizowane poprzez algorytm A*, który na bazie grafu jest w stanie odnaleźć drogę między dwoma zadanymi węzłami. W grze symulacyjnej graf zawiera węzły, które mogą mieć przypisany jeden z dwóch stanów: otwarty lub zamknięty. Wyliczona ścieżka nigdy nie prowadzi przez węzły zamknięte.

Algorytm został opisany przez Petera Harta, Nilsa Nilssona oraz Bertrama Raphaela w 1968 roku i początkowo nosił nazwę \emph{A}. Jednakże ze względu na zastosowanie heurystyki, polegającej na przeszukiwaniu grafu z pierwszeństwem analizy węzłów obiecujących (tj. tych, które znajdują się bliżej węzła docelowego), nadano algorytmowi nazwę \emph{A*}.

W grze symulacyjnej wykorzystywana jest javascript'owa implementacja algorytmu A*, przygotowana przez Briana Grinsteada w formie biblioteki javascript-astar\cite{astarPage}. Inicjalizacja grafu oraz przykład wyszukiwania ścieżki zostały przedstawione w listingu \ref{astarCode}. W definicji grafu \emph{1} oznacza węzeł otwarty, a \emph{0} węzeł zamknięty. Podczas definiowania przez użytkownika konfiguracji symulacji, w miejscu gdzie jest postawiona ściana węzły grafu zmieniają swój typ na zamknięty. Wynikiem wykonania wyszukiwania jest uporządkowana lista węzłów, jakie jednostka musi odwiedzić w drodze do celu. Algorytm wyznaczania scieżki jest zaimplementowany w metodzie \emph{calculatePath} klasy Game.Entity.

\begin{table}
\begin{center}
\begin{lstlisting}
	var graph = new Graph([
		[1,1,1,1,1,1,1,1,1,1],
		[1,1,1,1,0,0,1,1,1,1],
		[1,1,1,1,0,0,1,1,1,1],
		[1,1,1,1,0,0,1,1,1,1],
		[1,1,1,1,1,1,1,1,1,1],
	]);
	var start = graph.nodes[0][0];
	var end = graph.nodes[9][4];
	var result = astar.search(graph.nodes, start, end);
 \end{lstlisting}
\caption {Inicjalizacja grafu 10x5 oraz wyszukiwanie ścieżki między węzłami}
\label{astarCode}
\end{center}
\end{table}

\subsection{Zauważanie przeciwnika}\label{detectionSubsection}

\begin{figure}
\begin{center}
	\includegraphics[width=80mm,height=53mm]{images/detection}
	\caption[Zauważanie przeciwnika]{Zauważanie przeciwnika: Najbliższym przeciwnikiem jednostki A jest jednostka B. Jednostka C nie jest analizowana, ponieważ znajduje się za ścianą. Zielona linia to linia sprawdzająca ew. kolizje dla jednostki A\label{detectionImage}}
\end{center}
\end{figure}

Algorytm zauważania przeciwnika jest zaimplementowany w metodzie \emph{closestSeenOpponent} klasy Game.Entity. Zwraca on referencję do najbliższego przeciwnika, będącego w zasięgu wzroku. Algorytm iteruje po liście przeciwników, dokonując szeregu sprawdzeń tylko dla tych jednostek, które jeszcze żyją. Tzw. \emph{dłuższy dystans} obliczany jest między pozycją obserwującej jednostki a pozycją potencjalnego przeciwnika. \emph{Krótszy dystans} obliczany jest między końcem linii sprawdzającej ew. kolizje dla jednostki obserwującej a pozycją potencjalnego przeciwnika (rysunek \ref{detectionImage}). Jeżeli \emph{krótszy dystans} jest rzeczywiście mniejszy od \emph{dłuższego dystansu}, to oznacza to, że przeciwnik jest przed jednostką obserwującą\footnote{jednostki będące za plecami jednostki obserwującej są ignorowane}. Jeżeli dystans pomiędzy jednostkami jest większy od zasięgu wzroku lub jest większy niż zapamiętany dystans do aktualnego celu, to taki przeciwnik jest ignorowany. Jeżeli jednak odległości są mniejsze, a przeciwnik nie znajduje się za jakąkolwiek ścianą, to oznaczamy go za cel i zapamiętujemy nowy, aktualny dystans do celu. Pseudokod algorytmu jest zapisany w listingu \ref{detectionCode}.

\begin{table}
\begin{center}
\begin{lstlisting}
	cel = null
	dystans_do_celu = ja.zasieg_wzroku
	DLA KAZDEGO przeciwnik z lista_przeciwnikow WYKONUJ
		JEZELI przeciwnik.nie_zyje TO wykonaj_nastepna_iteracje
		dluzszy_dystans = oblicz_dystans (ja.pozycja, przeciwnik.pozycja)
		krotszy_dystans = oblicz_dystans (ja.koniec_promienia_kolizji, przeciwnik.pozycja)
		JEZELI krotszy_dystans < dluzszy_dystans ORAZ krotszy_dystans < dystans_do_celu TO
			JEZELI lista_scian_na_drodze (ja.pozycja, przeciwnik.pozycja) JEST PUSTA TO
				cel = przeciwnik
				dystans_do_celu = krotszy_dystans		
	ZWROC cel	
\end{lstlisting}
\caption {Pseudokod algorytmu zauważania przeciwnika}
\label{detectionCode}
\end{center}
\end{table}

\subsection{Podążanie jednostek w linii}

\begin{figure}
\begin{center}
	\includegraphics[width=80mm,height=53mm]{images/followEntity}
	\caption[Podążanie jednostek w linii]{Podążanie jednostek w linii: Jednostka A jest liderem. Z jednostką A, w określonym odstępie porusza się jednostka B, natomiast za jednostką B porusza się jednostka C\label{followEntityImage}}
\end{center}
\end{figure}

Antyterroryści w grze symulacyjnej poruszają się w szyku liniowym (rysunek \ref{followEntityImage}). Taka kolumna zaatakowana bezpośrednio od przodu lub od tyłu ma najmniejszą siłę ogniową. Linia zaatakowana od boku posiada bardzo dużą siłę ogniową, bowiem antyterroryści nie zasłaniają sobie na wzajem celu. Algorytm podążania w linii jest zaimplementowany w metodzie \emph{followEntity} klasy Game.Antiterrorist. Próbuje on dla danej jednostki znaleźć przyjazną jednostkę, która ją poprzedza i nie zginęła. Jeżeli taka jednostka nie zostanie znaleziona, to oznacza to, że dana jednostka jest liderem i należy zmienić jej stan na \emph{follow path}. W przeciwnym wypadku dana jednostka wylicza i podąża do współrzędnych celu, które są iloczynem odległości podążania oraz różnicy aktualnej pozycji znalezionego sprzymierzeńca i jego wektora prędkości. Pseudokod algorytmu jest zapisany w listingu \ref{followEntityCode}.

\begin{table}
\begin{center}
\begin{lstlisting}
	indeks = ja.indeks_w_grupie
	POWTARZAJ
		indeks = indeks - 1
		sprzymierzeniec = sprzymierzency[indeks]
	DOPOKI (ISTNIEJE(sprzymierzeniec) ORAZ sprzymierzeniec.nie_zyje)
	JEZELI (NIE_ISTNIEJE(sprzymierzeniec)) TO
		ja.jestLiderem = PRAWDA
		ja.zmien_stan('follow path')
	WPP
		ja.jednostka_cel = sprzymierzeniec

	ja.pozycja_celu = (sprzymierzeniec.wektor_pozycji - sprzymierzeniec.wektor_predkosci) * ja.odleglosc_podazania
	ja.seek()
\end{lstlisting}
\caption {Pseudokod algorytmu podążania za jednostką}
\label{followEntityCode}
\end{center}
\end{table}

\subsection{Atakowanie jednostki}
Atak na wrogą jednostkę jest poprzedzany obserwacją, która przeprowadzana jest niezależnie od stanu, w jakim znajduje się aktualnie jednostka. Z tego algorytmu korzystają zarówno antyterroryści, jak i terroryści. Jest on zaimplementowany w metodzie \emph{watchForEnemy} klasy Game.Entity. Na początku metoda korzysta z algorytmu zauważania przeciwnika (rozdział \ref{detectionSubsection}). Jeżeli dana jednostka nie widzi w danym momencie potencjalnego przeciwnika, a obecnie znajduje się w stanie \emph{attack}, to następuje przejście do stanu domyślnego tej jednostki. Jednak gdy przeciwnik został znaleziony, ale nie był wcześniej obserwowany, to dana jednostka rozpoczyna jego obserwację ustawiając czas do ataku na zdefiniowany w parametrach czas reakcji. Gdy jednak znaleziony przeciwnik jest już obserwowany i upłynie czas do ataku, to dana jednostka przechodzi do stanu \emph{attack}, jeżeli tylko nie ma przed sobą żadnych sprzymierzeńców stojących na linii ognia. Pseudokod algorytmu jest zapisany w listingu \ref{watchForEnemyCode}.

\begin{table}
\begin{center}
\begin{lstlisting}
	najblizszy_przeciwnik = znajdz_najblizszego_przeciwnika();
	JEZELI (NIE_ISTNIEJE(najblizszy_przeciwnik)) TO
		JEZELI (ja.aktualny_stan == 'attack')
			ja.pozycja_celu = null
			ja.zmien_stan_na_domyslny()
	WPP
		JEZELI (najblizszy_przeciwnik != ja.obserwowany_przeciwnik) TO
			ja.obserwowany_przeciwnik = najblizszy_przeciwnik
			ja.czas_do_ataku = ja.czas_reakcji
		WPP
			ja.czas_do_ataku = ja.czas_do_ataku - 1
			JEZELI (ja.czas_do_ataku < 0) TO
				JEZELI lista_sprzymierzencow_na_drodze (ja.pozycja, najblizszy_przeciwnik.pozycja) JEST PUSTA TO
					ja.jednostka_cel = najblizszy_przeciwnik
					ja.zmien_stan('attack')									
\end{lstlisting}
\caption {Pseudokod algorytmu obserwowania wroga}
\label{watchForEnemyCode}
\end{center}
\end{table}

Realizacja ataku polega na wystrzeliwaniu pocisku co określony interwał czasowy. Jeżeli upływa czas do kolejnego strzału, to tworzona jest nowa instancja pocisku, którego pozycja i orientacja są zgodne z odpowiednikami u strzelca. Obiekt pocisku wykonuje metodę \emph{move}, która sprawdza, czy pocisk nie trafił w ścianę lub jednostkę. W tym drugim przypadku zadawane są obrażenia zgodnie z energią, jaką posiadał pocisk w momencie trafienia. Pseudokod algorytmu ataku jest zapisany w listingu \ref{attackCode}.

\begin{table}
\begin{center}
\begin{lstlisting}
	ja.uaktualnij_pozycje_cel()
	ja.seek()				
	JEZELI (ja.czas_do_strzalu < 0) TO
		ja.czas_do_strzalu = ja.czas_miedzy_strzalami
		strzelec = ja
		UTWROZ('pocisk', strzelec)
	ja.czas_do_strzalu = ja.czas_do_strzalu - 1;
\end{lstlisting}
\caption {Pseudokod algorytmu atakowania wroga}
\label{attackCode}
\end{center}
\end{table}

\section{Parametryzacja}

Antyterroryści i terroryści w grze symulacyjnej współdzielą część metod oraz parametrów, które dziedziczą z klasy Game.Entity. Jednakże część parametrów wpływających na przebieg symulacji ma u nich różne wartości. Stroną faworyzowaną są tutaj antyterroryści, u których zakłada się, że są lepiej wyposażeni (np. większa szybkostrzelność broni, kamizelki kuloodporne) oraz są lepiej wyszkoleni (np. szybciej potrafią zareagować na pojawienie się przeciwnika). Dzięki takiej parametryzacji mniej liczebny odział antyterrorystów ma wyrównane szanse walki z liczniejszymi terrorystami. Tabela \ref{parameters} przedstawia porównanie wartości wybranych parametrów jednostek.

\begin{table}
\begin{center}
\begin{tabular}{|p{0.5\textwidth} | p{0.2\textwidth} p{0.2\textwidth}|}
\hline
Parametr & Antyterrorysta & Terrorysta\\\hline
	healthPoints (punkty życia) & 150 & 100\\\hline
	reactionTimeMax (czas reakcji) & 20 & 25\\\hline
	shootInterval (czas między strzałami) & 10 & 15\\\hline
	checkDirectionTimeMax (czas na sprawdzenie kierunku) & 100 & 1000\\\hline
	maxSpeed (maksymalna prędkość) & 0.025 & 0.020\\
\hline
\end{tabular}
\caption {Porównanie parametrów antyterrorysty i terrorysty\label{parameters}}
\end{center}
\end{table} 



% Rozdział 4
\chapter{Sztuczna inteligencja - taktyki}\label{tacticts}
W grze symulacyjnej, będącej przedmiotem niniejszej pracy dyplomowej, terroryści oraz antyterroryści posiadają sztuczną inteligencję\footnote{charakterystyka modelu SI została przedstawiona w rozdziale \ref{aiModelInfo}}, która pozwala im na podejmowanie decyzji oraz poruszanie się. Interfejsem, z którego jednostki uczestniczące w symulacji czerpią wiedzę o świecie gry, jest obiekt Game. 

\section{Taktyka antyterrorystów}
Antyterroryści posiadają strategię, która nakazuje im posiadanie lidera przez cały czas trwania symulacji. Lider antyterrorystów jest jednostką, za którą w szyku poruszają się pozostali antyterroryści. Jeżeli lider zginie, to natychmiastowo wybierany jest nowy lider, a działania grupy są kontynuowane.

Antyterrorysta posiada skończony zbiór stanów (rysunek \ref{atTacticImage}), które odzwierciedlają podjęte przez niego decyzje i definiują jego działania. Po zainicjalizowaniu obiektu antyterrorysty przechodzi on do stanu \emph{follow entity}, który pozwala mu podążać za swoimi poprzednikami. Jest to domyślny stan, do którego antyterrorysta może wrócić ze stanów, do których przeszedł na podstawie zdarzenia. Jeżeli antyterrorysta jest liderem, to następuje natychmiastowe przejście do stanu \emph{follow path}, które definiuje konieczność poruszania się po wyznaczonej ścieżce do następnego punktu kluczowego. Wraz z dotarciem do danego punktu kluczowego, wyznaczana jest ściseżka do następnego punktu kluczowego. Jeżeli antyterrorysta lider dotarł do ostatniego punktu kluczowego, to zmienia on swój stan na \emph{follow extraction}, który nakazuje jednostce poruszanie się po ścieżce do punktu startowego / końcowego antyterrorystów. Po dotarciu do tego punktu antyterrorysta zatrzymuje się i przechodzi do stanu bezczynności - \emph{idle}.

\begin{figure}
\begin{center}
	\includegraphics[width=160mm,height=105mm]{images/atTactic}
	\caption{Diagram przejść międzystanowych antyterrorysty\label{atTacticImage}}
\end{center}
\end{figure}

Każdy antyterrorysta może zmienić swój stan na podstawie zaistniałego zdarzenia. Postrzelenie antyterrorysty lub natrafienie przez niego na ścianę, wiąże się z~natychmiastowym wywołaniem stanu \emph{check location}. Stan ten definiuje poruszanie się jednostki do zadanego punktu, z wykorzystaniem wyliczonej ścieżki bezkolizyjnej. Antyterrorysta opuszcza ten stan po dotarciu do celu lub po upłynięciu limitu czasowego na wykonanie tej czynności. Zdarzenie polegające na zauważeniu przeciwnika, wywołuje stan \emph{attack}. Stan ten pozwala antyterroryście na oddawanie strzałów w kierunku zauważonego terrorysty, jeżeli na linii strzału nie znajduje się żaden antyterrorysta. Wyjście z tego stanu następuje po zabiciu terrorysty lub po straceniu celu z pola widzenia.

\section{Taktyka terrorystów}
Terroryści nie posiadają grupowej strategii działania, kierują się wyłącznie indywidualnie podejmowanymi decyzjami. Terrorysta posiada skończony zbiór stanów (rysunek \ref{terTacticImage}). Po zainicjalizowaniu obiektu terrorysty przechodzi on do stanu \emph{wander}, który pozwala mu na wędrowanie po świecie gry. Podczas wędrowania terrorysta może podjąć decyzję o~wykonaniu postoju, co wiąże się z przejściem do stanu \emph{stand}. Stan ten zatrzymuje jednostkę oraz odlicza czas pozostały do końca postoju, a po jego upłynięciu zmienia stan terrorysty ponownie na \emph{wander}.

\begin{figure}
\begin{center}
	\includegraphics[width=160mm,height=84mm]{images/terTactic}
	\caption{Diagram przejść międzystanowych terrorysty\label{terTacticImage}}
\end{center}
\end{figure}

Każdy terrorysta także może zmienić swój stan na podstawie zaistniałego zdarzenia. Postrzelenie terrorysty lub usłyszenie przez niego odgłosu wystrzału skutkuje wywołaniem stanu \emph{check location}, który definiuje identyczne zachowanie i warunek wyjścia ze stanu, jak w przypadku antyterrorysty. Podobnie jest ze zdarzeniem polegającym na zauważeniu przeciwnika. Wywołuje ono stan \emph{attack}, który nakazuje terroryście strzelać do wrogiej jednostki. Warunek wyjścia z tego stanu jest taki sam, jak u antyterrorysty.

Zdarzeniem wspólnym dla antyterrorystów oraz dla terrorystów jest otrzymanie śmiertelnego trafienia. W tym przypadku natychmiastowo zmieniany jest stan jednostki na bezczynność - \emph{idle}. Jednostka pozostająca w stanie bezczynności nie reaguje na zdarzenia, co skutkuje brakiem możliwości zmiany swego stanu.


\section{Opis algorytmów}
W tym rozdziale zostaną przedstawione wybrane algorytmy, jakie są wykorzystywane w grze symulacyjnej, będącej przedmiotem tej pracy dyplomowej. Słowny opis jest uzupełniony pseudokodami lub implementacją w języku Javascript.

\subsection{Myślenie i poruszanie się jednostek}
Antyterroryści i terroryści w grze symulacyjnej posiadają zaimplementowaną metodę \emph{think} pozwalającą na realizację działań zapisanych w konkretnych stanach (listing \ref{thinkAt} oraz \ref{thinkTer}). Prócz wywołania metod odpowiadającym konkretnym stanom, wywoływane są także metody \emph{watchForEnemy} oraz \emph{checkForCollision}. Ta pierwsza pozwala jednostce na obserwowanie otoczenia w poszukiwaniu przeciwników, natomiast druga na bezpieczne omijanie ścian. Metody odpowiadające za realizację poszczególnych stanów podejmują najczęściej decyzję, gdzie leży następny cel jednostki.

\begin{table}
\begin{center}
\begin{lstlisting}
    think: function(){
        this.watchForEnemy();
        switch(this.currentState) {
            case 'idle': break;
            case 'init': this.setup(); break;
            case 'follow entity': this.followEntity(); break;
            case 'follow path': this.followPath(); break;
            case 'follow extraction': this.followExtraction(); break;
            case 'check location': this.checkLocation(); break;
            case 'attack': this.attack(); break;
            default: this.changeToDefaultState(); break;
        }
        if (this.avoiding) this.wanderOrientation = this.getRotation();
        this.avoiding = this.checkForCollision();
    }
 \end{lstlisting}
\caption {Metoda think w klasie Game.Antiterrorist}
\label{thinkAt}
\end{center}
\end{table}

\begin{table}
\begin{center}
\begin{lstlisting}
    think: function(){
        this.watchForEnemy();
        switch(this.currentState) {
            case 'idle': break;
            case 'init': this.setup(); break;
            case 'stand': this.stand(); break;
            case 'wander': this.wander(); break;
            case 'check location': this.checkLocation(); break;
            case 'attack': this.attack(); break;
            default: this.changeToDefaultState(); break;
        }
        if (this.avoiding) this.wanderOrientation = this.getRotation();
        this.avoiding = this.checkForCollision();
    }
 \end{lstlisting}
\caption {Metoda think w klasie Game.Terrorist}
\label{thinkTer}
\end{center}
\end{table}

Gdy cel jest już zdefiniowany, to wykonywane są algorytmy niższego poziomu, odpowiedzialne za obliczenie wektora prędkości oraz prędkości, z jaką ma poruszać się jednostka. Takimi algorytmami są \emph{seek} (ruch w kierunku celu) oraz \emph{flee} (ruch w przeciwnym kierunku do celu). Metoda think jest wywoływana wewnątrz metody \emph{update} (listing \ref{update}), która służy do zmiany pozycji obiektu na scenie. Po wykonaniu metody \emph{think} zmieniana jest aktualna pozycja obiektu o wyliczony wektor prędkości, natomiast orientacja obiektu jest uaktualniana, by była zgodna z wektorem prędkości. Aktualizacja pozycji i orientacji nie następuje, gdy gra jest w trybie pauzy lub dany obiekt ruchomy już nie żyje.

\begin{table}
\begin{center}
\begin{lstlisting}
    update: function(frame) {
        if (!Game.paused && this.isAlive) {
            this.think();
            if (this.hasVelocity()) {
                this._updateCollisionRay();
                var pos = this.getVecPosition().add(this.getVecVelocity().multiply(frame.timeDiff * this.speed));
                this.setPosition(pos.e(1),pos.e(2));
                var rot = Math.atan2(-this.getVecVelocity().e(1), this.getVecVelocity().e(2));
                this.setRotation(rot);
            }
        }
   }
 \end{lstlisting}
\caption {Metoda update w klasie Game.Entity}
\label{update}
\end{center}
\end{table}

Metoda update jest wywoływana w anonimowej funkcji, przypisanej do zdarzenia zmiany klatki animacji\footnote{definicja obsługi tego zdarzenia znajduje się w metodzie Game.init()}. Obsługa tego zdarzenia stanowi główną pętlę aplikacji. Co klatkę następuje iteracja po wszystkich warstwach przypisanych do sceny (listing \ref{onFrame}). Dla każdej warstwy interujemy po wszystkich obiektach, jakie zostały dodane do danej warstwy. Jeżeli obiekt posiada zdefiniowaną metodę update (co oznacza, że jego klasa dziedziczy z Game.Entity), to jest ona wykonywana. Po przetworzeniu wszystkich obiektów w danej warstwie, wykonywana jest metoda draw, która renderuje ponownie warstwę na scenie.

\begin{table}
\begin{center}
\begin{lstlisting}
    this.stage.onFrame(function(frame) {
          for(var layerIndex in self.stage.getChildren()) {
              var layer = self.stage.getChildren()[layerIndex];
              for (var objectIndex in layer.getChildren()) {
                  var object = layer.getChildren()[objectIndex];
                  if(object.update) object.update(frame);
              }
              layer.draw();
          }
    });
 \end{lstlisting}
\caption {Rysowanie klatek animacji}
\label{onFrame}
\end{center}
\end{table}

\subsection{Wyznaczanie ścieżki - A*}\label{graph}
Wyznaczanie ścieżek bezkolizyjnych (uwzględniających położenie ścian na mapie) jest realizowane poprzez algorytm A*, który na bazie grafu jest w stanie odnaleźć drogę między dwoma zadanymi węzłami. W grze symulacyjnej graf zawiera węzły, które mogą mieć przypisany jeden z dwóch stanów: otwarty lub zamknięty. Wyliczona ścieżka nigdy nie prowadzi przez węzły zamknięte.

Algorytm został opisany przez Petera Harta, Nilsa Nilssona oraz Bertrama Raphaela w 1968 roku i początkowo nosił nazwę \emph{A}. Jednakże ze względu na zastosowanie heurystyki, polegającej na przeszukiwaniu grafu z pierwszeństwem analizy węzłów obiecujących (tj. tych, które znajdują się bliżej węzła docelowego), nadano algorytmowi nazwę \emph{A*}.

W grze symulacyjnej wykorzystywana jest javascript'owa implementacja algorytmu A*, przygotowana przez Briana Grinsteada w formie biblioteki javascript-astar\cite{astarPage}. Inicjalizacja grafu oraz przykład wyszukiwania ścieżki zostały przedstawione w listingu \ref{astarCode}. W definicji grafu \emph{1} oznacza węzeł otwarty, a \emph{0} węzeł zamknięty. Podczas definiowania przez użytkownika konfiguracji symulacji, w miejscu gdzie jest postawiona ściana węzły grafu zmieniają swój typ na zamknięty. Wynikiem wyszukiwania jest uporządkowana lista węzłów, jakie jednostka musi odwiedzić w drodze do celu. Algorytm wyznaczania scieżki jest zaimplementowany w metodzie \emph{calculatePath} klasy Game.Entity.

\begin{table}
\begin{center}
\begin{lstlisting}
	var graph = new Graph([
		[1,1,1,1,1,1,1,1,1,1],
		[1,1,1,1,0,0,1,1,1,1],
		[1,1,1,1,0,0,1,1,1,1],
		[1,1,1,1,0,0,1,1,1,1],
		[1,1,1,1,1,1,1,1,1,1],
	]);
	var start = graph.nodes[0][0];
	var end = graph.nodes[9][4];
	var result = astar.search(graph.nodes, start, end);
 \end{lstlisting}
\caption {Inicjalizacja grafu 10x5 oraz wyszukiwanie ścieżki między węzłami}
\label{astarCode}
\end{center}
\end{table}

\subsection{Zauważanie przeciwnika}\label{detectionSubsection}

\begin{figure}
\begin{center}
	\includegraphics[width=80mm,height=53mm]{images/detection}
	\caption[Zauważanie przeciwnika]{Zauważanie przeciwnika: Najbliższym przeciwnikiem jednostki A jest jednostka B. Jednostka C nie jest analizowana, ponieważ znajduje się za ścianą. Zielona linia to linia sprawdzająca ew. kolizje dla jednostki A\label{detectionImage}}
\end{center}
\end{figure}

Algorytm zauważania przeciwnika jest zaimplementowany w metodzie \emph{closestSeenOpponent} klasy Game.Entity. Zwraca on referencję do najbliższego przeciwnika, będącego w zasięgu wzroku. Algorytm iteruje po liście przeciwników, dokonując szeregu sprawdzeń tylko dla tych jednostek, które jeszcze żyją. Tzw. \emph{dłuższy dystans} obliczany jest między pozycją obserwującej jednostki a pozycją potencjalnego przeciwnika. \emph{Krótszy dystans} obliczany jest między końcem linii sprawdzającej ew. kolizje dla jednostki obserwującej a pozycją potencjalnego przeciwnika (rysunek \ref{detectionImage}). Jeżeli \emph{krótszy dystans} jest rzeczywiście mniejszy od \emph{dłuższego dystansu}, to oznacza to, że przeciwnik jest przed jednostką obserwującą\footnote{jednostki będące za plecami jednostki obserwującej są ignorowane}. Jeżeli dystans pomiędzy jednostkami jest większy od zasięgu wzroku lub jest większy niż zapamiętany dystans do aktualnego celu, to taki przeciwnik jest ignorowany. Jeżeli jednak odległości są mniejsze, a przeciwnik nie znajduje się za jakąkolwiek ścianą, to oznaczamy go za cel i zapamiętujemy nowy, aktualny dystans do celu. Pseudokod algorytmu jest zapisany w~listingu \ref{detectionCode}.

\begin{table}
\begin{center}
\begin{lstlisting}
	cel = null
	dystans_do_celu = ja.zasieg_wzroku
	DLA KAZDEGO przeciwnik z lista_przeciwnikow WYKONUJ
		JEZELI przeciwnik.nie_zyje TO wykonaj_nastepna_iteracje
		dluzszy_dystans = oblicz_dystans (ja.pozycja, przeciwnik.pozycja)
		krotszy_dystans = oblicz_dystans (ja.koniec_promienia_kolizji, przeciwnik.pozycja)
		JEZELI krotszy_dystans < dluzszy_dystans ORAZ krotszy_dystans < dystans_do_celu TO
			JEZELI lista_scian_na_drodze (ja.pozycja, przeciwnik.pozycja) JEST PUSTA TO
				cel = przeciwnik
				dystans_do_celu = krotszy_dystans		
	ZWROC cel	
\end{lstlisting}
\caption {Pseudokod algorytmu zauważania przeciwnika}
\label{detectionCode}
\end{center}
\end{table}

\subsection{Podążanie jednostek w linii}

\begin{figure}
\begin{center}
	\includegraphics[width=80mm,height=53mm]{images/followEntity}
	\caption[Podążanie jednostek w linii]{Podążanie jednostek w linii: Jednostka A jest liderem. Z jednostką A, w określonym odstępie porusza się jednostka B, natomiast za jednostką B porusza się jednostka C\label{followEntityImage}}
\end{center}
\end{figure}

Antyterroryści w grze symulacyjnej poruszają się w szyku liniowym (rysunek \ref{followEntityImage}). Taka kolumna zaatakowana bezpośrednio od przodu lub od tyłu ma najmniejszą siłę ogniową. Linia zaatakowana od boku posiada bardzo dużą siłę ogniową, bowiem antyterroryści nie zasłaniają sobie na wzajem celu. Algorytm podążania w linii jest zaimplementowany w metodzie \emph{followEntity} klasy Game.Antiterrorist. Próbuje on dla danej jednostki znaleźć przyjazną jednostkę, która ją poprzedza i nie zginęła. Jeżeli taka jednostka nie zostanie znaleziona, to oznacza to, że dana jednostka jest liderem i należy zmienić jej stan na \emph{follow path}. W przeciwnym wypadku dana jednostka wylicza i podąża do współrzędnych celu, które są iloczynem odległości podążania oraz różnicy aktualnej pozycji znalezionego sprzymierzeńca i jego wektora prędkości. Pseudokod algorytmu jest zapisany w~listingu \ref{followEntityCode}.

\begin{table}
\begin{center}
\begin{lstlisting}
	indeks = ja.indeks_w_grupie
	POWTARZAJ
		indeks = indeks - 1
		sprzymierzeniec = sprzymierzency[indeks]
	DOPOKI (ISTNIEJE(sprzymierzeniec) ORAZ sprzymierzeniec.nie_zyje)
	JEZELI (NIE_ISTNIEJE(sprzymierzeniec)) TO
		ja.jestLiderem = PRAWDA
		ja.zmien_stan('follow path')
	WPP
		ja.jednostka_cel = sprzymierzeniec

	ja.pozycja_celu = (sprzymierzeniec.wektor_pozycji - sprzymierzeniec.wektor_predkosci) * ja.odleglosc_podazania
	ja.seek()
\end{lstlisting}
\caption {Pseudokod algorytmu podążania za jednostką}
\label{followEntityCode}
\end{center}
\end{table}

\subsection{Atakowanie jednostki}
Atak na wrogą jednostkę jest poprzedzany obserwacją, która przeprowadzana jest niezależnie od stanu, w jakim znajduje się aktualnie jednostka. Z tego algorytmu korzystają zarówno antyterroryści, jak i terroryści. Jest on zaimplementowany w~metodzie \emph{watchForEnemy} klasy Game.Entity. Na początku metoda korzysta z~algorytmu zauważania przeciwnika (rozdział \ref{detectionSubsection}). Jeżeli dana jednostka nie widzi w~danym momencie potencjalnego przeciwnika, a obecnie znajduje się w stanie \emph{attack}, to następuje przejście do stanu domyślnego tej jednostki. Jednak gdy przeciwnik został znaleziony, ale nie był wcześniej obserwowany, to dana jednostka rozpoczyna jego obserwację ustawiając czas do ataku na zdefiniowany w parametrach czas reakcji. Gdy jednak znaleziony przeciwnik jest już obserwowany i upłynie czas do ataku, to dana jednostka przechodzi do stanu \emph{attack}, jeżeli tylko nie ma przed sobą żadnych sprzymierzeńców stojących na linii ognia. Pseudokod algorytmu jest zapisany w listingu \ref{watchForEnemyCode}.

\begin{table}
\begin{center}
\begin{lstlisting}
	najblizszy_przeciwnik = znajdz_najblizszego_przeciwnika();
	JEZELI (NIE_ISTNIEJE(najblizszy_przeciwnik)) TO
		JEZELI (ja.aktualny_stan == 'attack')
			ja.pozycja_celu = null
			ja.zmien_stan_na_domyslny()
	WPP
		JEZELI (najblizszy_przeciwnik != ja.obserwowany_przeciwnik) TO
			ja.obserwowany_przeciwnik = najblizszy_przeciwnik
			ja.czas_do_ataku = ja.czas_reakcji
		WPP
			ja.czas_do_ataku = ja.czas_do_ataku - 1
			JEZELI (ja.czas_do_ataku < 0) TO
				JEZELI lista_sprzymierzencow_na_drodze (ja.pozycja, najblizszy_przeciwnik.pozycja) JEST PUSTA TO
					ja.jednostka_cel = najblizszy_przeciwnik
					ja.zmien_stan('attack')									
\end{lstlisting}
\caption {Pseudokod algorytmu obserwowania wroga}
\label{watchForEnemyCode}
\end{center}
\end{table}

Realizacja ataku polega na wystrzeliwaniu pocisku co określony interwał czasowy. Jeżeli upływa czas do kolejnego strzału, to tworzona jest nowa instancja pocisku, którego pozycja i orientacja są zgodne z odpowiednikami u strzelca. Obiekt pocisku wykonuje metodę \emph{move}, która sprawdza, czy pocisk nie trafił w ścianę lub jednostkę. W tym drugim przypadku zadawane są obrażenia zgodnie z energią, jaką posiadał pocisk w momencie trafienia. Pseudokod algorytmu ataku jest zapisany w listingu \ref{attackCode}.

\begin{table}
\begin{center}
\begin{lstlisting}
	ja.uaktualnij_pozycje_cel()
	ja.seek()				
	JEZELI (ja.czas_do_strzalu < 0) TO
		ja.czas_do_strzalu = ja.czas_miedzy_strzalami
		strzelec = ja
		UTWROZ('pocisk', strzelec)
	ja.czas_do_strzalu = ja.czas_do_strzalu - 1;
\end{lstlisting}
\caption {Pseudokod algorytmu atakowania wroga}
\label{attackCode}
\end{center}
\end{table}

\subsection{Sprawdzanie lokacji}

W szczególnych przypadkach jednostki mogą przejść do stanu \emph{check location}, którego definicja nakazuje przejście jednostki do zadanego miejsca na mapie. Przed przejściem do tego stanu jest ustawiany cel i obliczana bezkolizyjna ścieżka do niego (listing \ref{checkLocationSet}). Dla antyterrorystów przejście do stanu \emph{check location} jest wywoływane w następujących sytuacjach:
\begin{itemize}
	\item antyterrorysta nie będący liderem natrafi na ścianę, która odgradza go od antyterrorysty poprzednika; wyznaczanym celem jest pozycja poprzednika na mapie
	\item antyterrorysta otrzyma trafienie; wyznaczanym celem jest pozycja strzelca na mapie
\end{itemize}

Terroryści również korzystają z implementacji stanu \emph{check location}. W ich przypadku stan ten jest wywoływany gdy:
\begin{itemize}
	\item terrorysta usłyszy odgłos wystrzału lub trafienia; wyznaczanym celem jest pozycja pocisku w momencie wystrzału lub trafienia
	\item antyterrorysta otrzyma trafienie; wyznaczanym celem jest pozycja strzelca na mapie
\end{itemize}

Jednostki posiadają ograniczony czas na przejście do zadanej lokacji (listing \ref{checkLocationDo}). Antyterroryści posiadają bardzo niską wartość maksymalnego czasu na sprawdzenie lokacji, gdyż priorytetem dla nich jest wykonanie planu poruszając się w grupie, gdzie posiadają większą siłę ognia. Warunkiem wyścia ze stanu jest upłynięcie czasu na sprawdzenie lokacji lub dotarcie do niej. Jednostka opuszczająca stan \emph{check location} przechodzi do własnego domyślnego stanu.

\begin{table}
\begin{center}
\begin{lstlisting}
	ja.czas_na_sprawdzenie_lokacji = ja.maksymalny_czas_na_sprawdzenie_lokacji
	ja.wytycz_sciezke_do(cel)
	ja.zmien_stan('check location')
\end{lstlisting}
\caption {Pseudokod algorytmu sprawdzania lokacji (metoda setCheckLocation)}\label{checkLocationSet}
\label{attackCode}
\end{center}
\end{table}


\begin{table}
\begin{center}
\begin{lstlisting}
	wezel = sciezka[ja.indeks_wezla]
	JEZELI (ja.czas_na_sprawdzenie_lokacji < 0 LUB NIE_ISTNIEJE(wezel)) TO
		ja.zmien_stan_na_domyslny()	
	WPP
		ja.pozycja_celu = wspolrzedne_wezla(wezel)
		JEZELI (ja.dotarl_do_celu) TO
			ja.indeks_wezla = ja.indeks_wezla + 1
		ja.czas_na_sprawdzenie_lokacji = ja.czas_na_sprawdzenie_lokacji - 1
		ja.seek()	
\end{lstlisting}
\caption {Pseudokod algorytmu sprawdzania lokacji (metoda checkLocation)}\label{checkLocationDo}
\label{attackCode}
\end{center}
\end{table}


% Zakończenie
\newpage
\phantomsection \label{zakonczenieChapter}
\addcontentsline{toc}{chapter}{Zakończenie}
\chapter*{Zakończenie}
Realizacja projektu gry symulacyjnej, która jest przedmiotem tej pracy dyplomowej, rozpoczęła się od pomysłu wizualizacji odpowiednio uproszczonego procesu planowania operacji antyterrorystycznej. Po przygotowaniu pierwszej wersji szkicu interfejsu i rozpisaniu scenariuszy, jakie powinny zostać zaimplementowane w~pierwszym etapie, należało wybrać technologię, w jakiej gra będzie implementowana. Pierwszy prototyp powstał w języku Ruby z wykorzystaniem biblioteki Rubygame\footnote{strona projektu Rubygame - http://rubygame.org/}. Jednakże końcowy wybór padł na Javascript i HTML5 Canvas, które w tym wypadku są wspierane przez bibliotekę Kinetic.js. Na korzyść tych technologi przemawiało kilka faktów: 
\begin{itemize}
	\item Javascript posiada dużo mniej wymagające środowisko uruchomieniowe od Ruby'iego
	\item Kinetic.js jest stale rozwijaną biblioteką w przeciwieństwie to Rubygame
	\item implementacja interfejsu w HTML jest bardzo prosta
\end{itemize}

Implementacja kolejnych etapów następowała sprawnie. Informacja zwrotna uzyskiwana podczas testów przyczyniała się do dodania nowych funkcjonalności oraz motywowała do dalszej pracy. Każdy etap dostarczał nową, kompletną funkcjonalność. Gra symulacyjna może być nadal rozwijana, a najbliższe działania powinny się skupić na następujących obszarach:
\begin{itemize}
	\item Urozmaicenie taktyk terrorystów: terroryści mogą z określonym prawdopodobieństwem uciekać na widok antyterrorystów
	\item Urozmaicenie taktyk antyterrorystów: podczas ataku antyterroryści zmieniają szyk na tyralierę, by dysponować większą siłą ogniową w kierunku celu
	\item Refaktoring: wydzielenie części kodu Game.Entity do nowej klasy Game.Person
	\item Interfejs: przygotowanie atrakcyjniejszej oprawy graficznej
	\item Testy: sporządzenie testów automatycznych, weryfikujących poprawność działania interfejsu oraz ważniejszych metod publicznych
\end{itemize}

Wykorzystane technologie w zupełności wystarczyły do zaimplementowania projektu. Javascript i HTML5 Canvas można z powodzeniem wykorzystywać do wykonywania prostych, nie pochłaniających dużej ilości zasobów (CPU, Ram) aplikacji. Z drugiej strony, wraz z obserwowanym, bardzo szybkim tempem rozwoju technologi internetowych (np. gry 3D w przeglądarce z użyciem WebGL), możemy wkrótce się doczekać bardzo atrakcyjnych i wydajnych rozwiązań, które będą kreować nowe standardy w dziedzinie tworzenia gier wideo.


% Literatura:
\newpage
\phantomsection \label{bibliografiaChapter}
\addcontentsline{toc}{chapter}{Bibliografia}
\bibliographystyle{unsrt}
\nocite{*}
\bibliography{chapters/bibliografia}

% Spisy tabel i rysunków:
\newpage
\phantomsection \label{tabeleChapter}
\addcontentsline{toc}{chapter}{\listtablename}
\listoftables

\newpage
\phantomsection \label{rysunkiChapter}
\addcontentsline{toc}{chapter}{\listfigurename}
\listoffigures

\end{document}
